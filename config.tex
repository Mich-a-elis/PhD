\usepackage[a4paper,left=25mm,right=25mm,bottom=1.5cm,top=15mm,headheight=27.12pt,footskip=0.5cm,includehead,includefoot,bindingoffset=0pt]{geometry}
\usepackage[cleardoublepage=empty]{scrextend}
\usepackage{pdfpages}
\usepackage[T1]{fontenc}
\usepackage[french]{babel}
\usepackage{minitoc}
\mtcselectlanguage{french}
\usepackage{graphicx}
\usepackage{amsmath,amstext,amssymb}
\usepackage{stmaryrd}
\usepackage{amsthm}
\usepackage{tikz}
\usepackage{mathtools}
%\usepackage{algorithm}
%\usepackage{algorithmic}
%\usepackage[noend]{algpseudocode}
%\usepackage{algorithm2e}
\usepackage{array}
\usepackage{tabularx}
\usepackage{imakeidx}
\usepackage{float}
\usepackage[framemethod=TikZ]{mdframed}
\usepackage[intoc]{nomencl}
\usepackage[hidelinks]{hyperref}
\usepackage{cleveref}
\usepackage{fancyhdr}
\usepackage{appendix}
\usepackage{stackengine}
\usepackage{scalerel}
\usepackage{verbatimbox}
\usepackage{fourier-orns}
%\usepackage{frbib}
\usepackage[most]{tcolorbox}
\usepackage[toc,acronym,nonumberlist]{glossaries}
\hypersetup{colorlinks,
    linkcolor={blue!50!black},
    citecolor={green!50!black},
    urlcolor={red!80!black}
}
\usepackage{tikz}
\usepackage{pgfplots,pgfplotstable}
\usepgfplotslibrary{groupplots}
\pgfplotsset{compat=1.16}
\usetikzlibrary{positioning}
\usetikzlibrary{calc,spy,shapes,patterns} 
\usepackage{caption}
\usepackage{subcaption}
\usepackage[nottoc]{tocbibind}
\usepackage{multirow}
\usepackage{array}
\usepackage{bbold}

\usepackage{algorithm2e}
\renewcommand{\listalgorithmcfname}{Liste des Algorithmes}%
\renewcommand{\algorithmcfname}{Algorithme}%
\renewcommand{\algorithmautorefname}{algorithme}%
\renewcommand{\algorithmcflinename}{ligne}%
%\renewcommand{\algocf@typo}{\ }%
%\renewcommand{\@algocf@procname}{Proc\'edure}%
%\renewcommand{\@algocf@funcname}{Fonction}%
\renewcommand{\procedureautorefname}{proc\'edure}%
\renewcommand{\functionautorefname}{fonction}%
%\renewcommand{\algocf@languagechoosen}{french}%

%\SetKwHangingKw{HDonnees}{Donnees$\rightarrow$}
%\SetKwInput{Donnees}{Donn\'ees}%
%\SetKwInput{Res}{R\'esultat}%
\SetKwInput{Entree}{Entr\'ees}%
\SetKwInput{Sortie}{Sortie}%
\SetKwInput{Sorties}{Sorties}%
%\SetKw{KwA}{\`a}%
%\SetKw{Retour}{retourner}%
%\SetKwBlock{Deb}{d\'ebut}{fin}%
\SetKwRepeat{Repeter}{r\'ep\'eter}{jusqu'\`a}%
%
\SetKwIF{Si}{SinonSi}{Sinon}{si}{alors}{sinon si}{sinon}{fin si}%
\SetKwSwitch{Suivant}{Cas}{Autre}{suivant}{faire}{cas o\`u}{autres cas}{fin cas}{fin d'alternative}%
\SetKwFor{Pour}{pour}{faire}{fin pour}%
\SetKwFor{PourPar}{pour}{faire en parall\`ele}{fin pour}%
\SetKwFor{PourCh}{pour chaque}{faire}{fin pour chaque}%
\SetKwFor{PourTous}{pour tous}{faire}{fin pour tous}%
\SetKwFor{Tq}{tant que}{faire}{fin tq}%



\newcommand{\introchap}[1]{%
	\chapter*{#1}
	\addstarredchapter{#1}
	\markboth{\uppercase{#1}}{}		
}

\newcommand{\prodscal}[2]{\left\langle#1,#2\right\rangle} % produit scalaire
\newcommand{\norme}[1]{\left\lVert#1\right\rVert} %norme
\newcommand{\abs}[1]{\left\lvert#1\right\rvert} %abs
\newcommand{\enstq}[2]{\left\{#1\mathrel{}\middle|\mathrel{}#2\right\}} %ensemble tel que
\newcommand\V[1]{\overrightarrow{#1}} %vecteur

\sloppy % pour que les mots ne depassent pas en fin de ligne
\newcommand{\bleu}[1]{{\color{blue} #1 }}
\newcommand{\rouge}[1]{{\color{red} #1 }}

\usepackage{multirow}
\usepackage{array}
\newcommand{\ds}{\displaystyle}
\newcolumntype{M}[1]{>{\centering\arraybackslash}m{#1}}
\newcommand{\supp}{\mathrm{Supp}}
\setcounter{MaxMatrixCols}{20}

\addto\captionsfrench{\renewcommand{\bibname}{Références}}
\addto\captionsfrench{\renewcommand{\nomname}{Notations}}
\addtocounter{tocdepth}{3}
\setcounter{secnumdepth}{3}
% Exclure les paragraphes et les sous-sous-sections de la table des matières
\setcounter{tocdepth}{2}

%%%%% TIKZ
\usetikzlibrary{shadows,arrows,positioning}
% Define the layers to draw the diagram
\pgfdeclarelayer{background}
\pgfdeclarelayer{foreground}
\pgfsetlayers{background,main,foreground}

% Define block styles
\tikzstyle{materia}=[draw, fill=blue!20, text width=6.0em, text centered, minimum height=1.5em,drop shadow]
\tikzstyle{etape} = [materia, text width=8em, minimum width=10em,
  minimum height=3em, rounded corners, drop shadow]
  \tikzstyle{materia2}=[draw, fill=blue!20, text width=6.0em, text centered, minimum height=1.5em,drop shadow]
\tikzstyle{etapered} = [materia2, text width=8em, minimum width=10em,
  minimum height=3em, rounded corners, drop shadow]
\tikzstyle{texto} = [above, text width=6em, text centered]
\tikzstyle{linepart} = [draw, thick, color=black!50, -latex', dashed]
\tikzstyle{line} = [draw, thick, color=black!50, -latex']
\tikzstyle{ur}=[draw, text centered, minimum height=0.01em]

% Define distances for bordering
\newcommand{\blockdist}{1.3}
\newcommand{\edgedist}{1.5}

\newcommand{\etape}[2]{node (p#1) [etape]
  {#2}}

% Draw background
\newcommand{\background}[5]{%
  \begin{pgfonlayer}{background}
    % Left-top corner of the background rectangle
    \path (#1.west |- #2.north)+(-0.5,0.25) node (a1) {};
    % Right-bottom corner of the background rectanle
    \path (#3.east |- #4.south)+(+0.5,-0.25) node (a2) {};
    % Draw the background
    \path[fill=yellow!20,rounded corners, draw=black!50, dashed]
      (a1) rectangle (a2);
      \path (#3.east |- #2.north)+(0,0.25)--(#1.west |- #2.north) node[midway] (#5-n) {};
      \path (#3.east |- #2.south)+(0,-0.35)--(#1.west |- #2.south) node[midway] (#5-s) {};
      \path (#3.east |- #2.north)+(0.7,0)--(#3.east |- #4.south) node[midway] (#5-w) {};
  \end{pgfonlayer}}

% ------ ENVIRONNEMENTS ------%

\definecolor{bleudefrance}{rgb}{0.19, 0.55, 0.91}
\definecolor{vertolive}{rgb}{0.33, 0.42, 0.18}
\definecolor{bluepy}{HTML}{1F77B4}
\definecolor{orangepy}{HTML}{FF7F0E}
\definecolor{greenpy}{HTML}{2CA02C}
\definecolor{redpy}{HTML}{D62728}

\def \ifempty#1{\def\temp{#1} \ifx\temp\empty }
\def \nomtheoreme#1{\ifempty{#1} \hspace{-5mm} \else\normalfont #1 \fi}

\newtcbtheorem[number within=chapter]{Prop}{}{
        enhanced,
        rounded corners,
        attach boxed title to top left={
            xshift=10mm,
            yshift=-3.5mm,
            yshifttext=1mm
        },
        top=1.5em,
        coltitle=black,
        colback=white,
        colframe=red,
        fonttitle=\bfseries,
        boxed title style={
            rounded corners,
            size=small,
            colback=white!20,
            colframe=red,
        } 
    }{thm}
    
\newtcbtheorem[number within=chapter]{Definition}{}{
        enhanced,
        rounded corners,
        attach boxed title to top left={
            xshift=10mm,
            yshift=-3.5mm,
            yshifttext=1mm
        },
        top=1.5em,
        coltitle=black,
        colback=white,
        colframe=bleudefrance,
        fonttitle=\bfseries,
        boxed title style={
            rounded corners,
            size=small,
            colback=white!20,
            colframe=bleudefrance,
        }  
    }{def}

\newtheorem{definition}{Définition}[chapter]
\newtheorem{theorem}[definition]{Théorème}
\newtheorem{remark}[definition]{Remarque}
\newtheorem{proposition}[definition]{Proposition}
\newtheorem{corollary}[definition]{Corollaire}
\newtheorem{lemma}[definition]{Lemme}
\newtheorem{example}[definition]{Exemple}
%\newtheorem{algorithm}{Algorithme}[chapter]
\addto\captionsfrench{\def\proofname{Preuve}}
\renewcommand\qedsymbol{$\blacksquare$}

\renewcommand{\thefootnote}{\alph{footnote}}


\newlength\glyphwidth
\newcommand\talldot{%
\ThisStyle{%
%%%% 1ST ARGUMENT OF \stackengine (GAP BETWEEN STACKED DOTS)
  \stackengine{.0ex}{$\SavedStyle.$}{$\SavedStyle.$}{O}{c}{F}{F}{S}%
}}
\newsavebox\hatglyphCONTENT

\newcommand\hatglyph{\ThisStyle{\scalerel*{\usebox{\hatglyphCONTENT}}%
                     {\SavedStyle\talldot}}}
\newcommand\shifthat[2]{%
  \ThisStyle{%
%%%% 1ST ARGUMENT OF \stackengine (GAP BETWEEN GLYPH AND \hatglyph)
    \stackengine{0.0ex}{\(\SavedStyle#2\)}%
                {\(\rule{#1}{0ex}\SavedStyle\hatglyph\)}{O}{c}{F}{T}{S}%
  }%
}
\renewcommand\widehat[1]{%
  \ThisStyle{\setlength\glyphwidth{\widthof{$\SavedStyle x$}}%
    \if A#1\shifthat{0.5\glyphwidth}{#1}\else
    \if B#1\shifthat{0.2\glyphwidth}{#1}\else
    \if x#1\shifthat{0.1\glyphwidth}{#1}\else
    \shifthat{0.15\glyphwidth}{#1}% THIS IS THE DEFAULT VALUE
    \fi
    \fi
    \fi
  }%
}

\def\Dcalb{\bm{\mathcal D}}
\def\Gcalb{\bm{\mathcal G}}
\def\Kcalb{\bm{\mathcal K}}
\def\Scalb{\bm{\mathcal S}}
\def\Tcalb{\bm{\mathcal T}}

\def\E{\textit{\textbf{E}}\,}
\def\H{\textit{\textbf{H}}\,}
\def\Ec{\displaystyle\widehat{\E}}
\def\Hc{\displaystyle\widehat{\H}}
\def\n{\textbf{n}}
\def\Ei{\ensuremath{\Eb^{\,i}}}
\def\Hi{\ensuremath{\Hb^{\,i}}}
\def\Ed{\ensuremath{\Eb^{\,d}}}
\def\Hd{\ensuremath{\Hb^{\,d}}}
\def\Re{\text{Re}}
\def\eR{\mathbb{R}}
\def\AS{\backslash}
\renewcommand{\bar}[1]{\overline{#1}}
\def\e0{\epsilon_0}
\def\m0{\mu_0}
\def\w{\omega}
\def\J{\textit{\textbf{J}}\,}
\def\M{\textit{\textbf{M}}\,}

% ====== COULEURS ====== %

\definecolor{bleudefrance}{rgb}{0.19, 0.55, 0.91}
\definecolor{vertolive}{rgb}{0.33, 0.42, 0.18}
\definecolor{bluepy}{HTML}{1F77B4}
\definecolor{orangepy}{HTML}{FF7F0E}

% ====== POLICE ====== %

\newcommand{\It}[1]{\textit{#1}}
\newcommand{\Bf}[1]{\textbf{#1}}

% ====== LETTRES CAPITALES ====== %

\def\A{{A}}
\def\B{{B}}
\def\C{{C}}
\def\D{{\text{D}}}
\def\E{{E}}
\def\F{{F}}
\def\G{{G}}
\def\H{{H}}
\def\I{{I}}
\def\J{{\text{J}}}
\def\K{{K}}
\def\L{{L}}
\def\M{{M}}
\def\N{{\text{N}}}
\def\O{{O}}
\def\P{{\text{P}}}
\def\Q{{Q}}
\def\R{{R}}
\def\S{{\text{S}}}
\def\T{{\text{T}}}
\def\U{{U}}
\def\V{{\text{V}}}
\def\W{{W}}
\def\X{{X}}
\def\Y{{\text{Y}}}
\def\Z{{\text{Z}}}

% ====== LETTRES MINUSCULES ====== %

\def\j{{\text{j}}}
\def\v{{\text{v}}}

% ====== LETTRES CAPITALES GRAS ====== %

\def\Eb{{\bf E}}
\def\Hb{{\bf H}}
\def\Jb{{\bf J}}
\def\Mb{{\bf M}}
\def\phib{\boldsymbol{\phi}}
\def\Phib{\boldsymbol{\Phi}}
\def\psib{\boldsymbol{\psi}}
\def\Psib{\boldsymbol{\Psi}}

% ====== LETTRES MINUSCULES GRAS ====== %

\def\jb{{\bf j}}
\def\ub{{\bf u}}
\def\vb{{\bf v}}

% ====== LETTRES CALLIGRAPHIQUES ====== %

\def\Acal{{\mathcal A}}
\def\Bcal{{\mathcal B}}
\def\Ccal{{\mathcal C}}
\def\Dcal{{\mathcal D}}
\def\Ecal{{\mathcal E}}
\def\Fcal{{\mathcal F}}
\def\Gcal{{\mathcal G}}
\def\Hcal{{\mathcal H}}
\def\Ical{{\mathcal I}}
\def\Jcal{{\mathcal J}}
\def\Kcal{{\mathcal K}}
\def\Lcal{{\mathcal L}}
\def\Mcal{{\mathcal M}}
\def\Ncal{{\mathcal N}}
\def\Ocal{{\mathcal O}}
\def\Pcal{{\mathcal P}}
\def\Qcal{{\mathcal Q}}
\def\Rcal{{\mathcal R}}
\def\Scal{{\mathcal S}}
\def\Tcal{{\mathcal T}}
\def\Ucal{{\mathcal U}}
\def\Vcal{{\mathcal V}}
\def\Wcal{{\mathcal W}}
\def\Xcal{{\mathcal X}}
\def\Ycal{{\mathcal Y}}
\def\Zcal{{\mathcal Z}}

% ====== MATHÉMATIQUES ====== %

\newcommand{\thicktilde}[1]{\mathbf{\tilde{\text{$#1$}}}}
\newcommand{\interv}[1]{\ensuremath{[\![#1]\!]}}
\newcommand{\MI}[1]{\ensuremath{\hat{#1}}}
\newcommand{\pair}[3]{\ensuremath{\overline{#1_{#3}#2_{#3}}}}
\def\im{{\bf i}}
\newcommand{\FEV}{\ensuremath{\mathcal{H}}}
\newcommand{\FEM}{\ensuremath{\text{H}}}
\newcommand{\dual}[2]{\langle#1,#2\rangle}
\newcommand{\norm}[2]{\ensuremath{\|#1\|_{#2}}}
\def\ne{{\tiny{N}}}
\def\un{u_\ne}
\def\uh{{\tt u}}
\def\uhi{{\tt u}_i}
\def\xi{{X_i}}
\def\uhn{{\tt u}_{\ne}}
\def\uhni{{\tt u}_{\ne i}}
\def\uinc{u^{i}}
\def\varphibf{\bm{\varphi}}
\def\Hoij{\H_{1_{ij}}}
\newcommand{\strkron}{\ensuremath{\,\underline{\overline{\otimes}}\,}}
\newcommand{\contract}{\ensuremath{\bullet}}
\newcommand{\multind}[3]{\ensuremath{#1_{#2},\hdots,#1_{#3}}}
\newcommand{\Sum}[2]{\ensuremath{\sum\limits_{#1}^{#2}}}
\newcommand{\Prod}[2]{\ensuremath{\prod\limits_{#1}^{#2}}}
\def\cC{\mathbb{C}}
\def\cR{\mathbb{R}}
\def\cN{\mathbb{N}}
\def\cK{\mathbb{K}}
\def\mtimes{\times\cdots\times}
\def\reshape{\texttt{reshape}}
%\renewcommand{\algorithmicrequire}{\textbf{In :}}
%\renewcommand{\algorithmicensure}{\textbf{Out :}}
\newcommand{\zkron}{\ensuremath{\oslash}}
\def\maxvol{\texttt{maxvol}}
\def\rmaxvol{\texttt{rect_maxvol}}
\def\ttSVD{\texttt{TT-SVD}}
\def\ttCross{\texttt{TT-Cross}}
\def\rounding{\texttt{rounding}}
\def\kkt{\texttt{KKT}}
\def\vec{\text{vec}}
\newcommand{\matSubTT}[2]{\ensuremath{#1_{#2}}}
\newcommand{\subTT}[2]{\ensuremath{#1_{#2}}}
\newcommand{\frameMat}[2]{\ensuremath{#1_{\neq#2}}}
\def\amen{\texttt{AMEN}}
\def\mamen{\texttt{MAMEN}}
\def\inv{\texttt{AMEN-Inverse}}
\def\lrsweep{\texttt{leftToRightSweep}}
\def\rlsweep{\texttt{rightToLeftSweep}}
\newcommand{\modeUnfold}[2]{\ensuremath{#1^{(#2)}}}
\newcommand{\unfold}[2]{\ensuremath{\underline{#1}_{#2}}}
\def\qttMVPo{\texttt{QTT-MVP1}}
\def\qttMVPb{\texttt{QTT-MVP2}}
\def\bdmMvp{\texttt{BDM-MVP1}}
\def\bdmMvpb{\texttt{BDM-MVP2}}
\def\nlbda{\ensuremath{n_{\lambda}}}
\def\Sg{[\S]}
\def\Ng{[\N]}
\def\Dg{[\D]}
\def\Tg{[\T]}
\newcommand{\glk}[1]{[#1]}
\def\dx{\text{d}x}
\def\dy{\text{d}y}
\DeclareMathOperator*{\argmax}{arg\,max}
\DeclareMathOperator*{\argmin}{arg\,min}
\def\Id{\text{Id}}
\def\nx{\ensuremath{{\bf n} \times}}
\newcommand{\pow}[1]{^{#1}}

% ====== NOTATIONS ====== %

\def\Hmat{$\mathcal{H}$-matrix}
\def\Hmats{$\mathcal{H}$-matrices}
\def\cad{c'est-à-dire\,}
\def\id{c.-à-d.}
\def\CPUbuild{$\text{CPU}_{\footnotesize \sf build}$}
\def\CPUmvp{$\text{CPU}_{\footnotesize{\sf mvp}}$}
\def\MEMbuild{$\text{MEM}_{\footnotesize \sf build}$}
\def\CPUsolve{$\text{CPU}_{\footnotesize \sf solve}$}
\def\oeuvre{\oe uvre}
\def\ttVec{TT-tenseur}
\def\ttMat{TT-matrice}
\newcommand{\prc}{PREC}
\newcommand{\slie}{SIE}
\newcommand{\hie}{HIE}
\newcommand{\bdmo}{BDM1}
\newcommand{\bdmd}{BDM2}
\newcommand{\bdmt}{BDM3}
\newcommand{\bdmtb}{BDM3$_{\sf bis}$}

% ====== TIKZ ====== %

\DeclareRobustCommand{\legendsquare}[1]{%
  \tikz[baseline=(a.south)]{\node[#1, inner sep=.8ex, outer sep=0] (a) {};}%
}
\newcommand{\mysize}{0.8}
\newcommand{\Qgrid}[1]{
    \foreach \row in {1, ...,#1} {
    \draw ($(\row-1,0)$)--($(\row - 1,#1-1)$);
    \draw ($(0,\row-1)$)--($(#1-1,\row-1)$);
    }}
\newcommand{\Tgrid}{
    \foreach \row in {1, ...,5} {
    \draw ($(\row-1,0)$)--($(\row - 1,4)$);
    \draw ($(0,\row-1)$)--($(4,\row-1)$);
    }
    \foreach \row in {1, ...,4} {
    \foreach \col in {1,...,4}{
    \draw ($(\row-1,\col)$)--($(\row,\col-1)$);
    }}}    
\newcommand{\Cube}[5]{
    \draw[ultra thin,fill=#5]($(#1,#2,#3)$)--($(#1+#4,#2,#3)$)--($(#1+#4,#2+#4,#3)$)--($(#1,#2+#4,#3)$)--cycle;
    \draw[ultra thin,fill=#5]($(#1+#4,#2,#3)$)--($(#1+#4,#2,#3-#4)$)--($(#1+#4,#2+#4,#3-#4)$)--($(#1+#4,#2+#4,#3)$)--cycle;
    \draw[ultra thin,fill=#5]($(#1,#2+#4,#3)$)--($(#1,#2+#4,#3-#4)$)--($(#1+#4,#2+#4,#3-#4)$)--($(#1+#4,#2+#4,#3)$)--cycle;
    }    
    
\newcommand{\sizecirc}{\small}
\newcommand{\Square}[3]{
    \draw[line width=0.25mm]($(#1,#2)$)--($(#1+#3,#2)$)--($(#1+#3,#2+#3)$)--($(#1,#2+#3)$)--cycle;}
    
\newcommand{\SquareCirc}[5]{
    \draw[line width=0.01mm]($(#1,#2)$)--($(#1+#3,#2)$)--($(#1+#3,#2+#3)$)--($(#1,#2+#3)$)--cycle;
    \draw ($(#1+0.25*#3,#2+0.25*#3)$) node{{#5\color{#4}$\bullet$}};
    \draw ($(#1+0.5*#3,#2+0.25*#3)$) node{{#5\color{#4}$\bullet$}};
    \draw ($(#1+0.75*#3,#2+0.25*#3)$) node{{#5\color{#4}$\bullet$}};
    \draw ($(#1+0.25*#3,#2+0.5*#3)$) node{{#5\color{#4}$\bullet$}};
    \draw ($(#1+0.5*#3,#2+0.5*#3)$) node{{#5\color{#4}$\bullet$}};
    \draw ($(#1+0.75*#3,#2+0.5*#3)$) node{{#5\color{#4}$\bullet$}};
    \draw ($(#1+0.25*#3,#2+0.75*#3)$) node{{#5\color{#4}$\bullet$}};
    \draw ($(#1+0.5*#3,#2+0.75*#3)$) node{{#5\color{#4}$\bullet$}};
    \draw ($(#1+0.75*#3,#2+0.75*#3)$) node{{#5\color{#4}$\bullet$}};
    }
    
    
\newcommand{\SquareCircBis}[5]{
    \draw[line width=0.01mm]($(#1,#2)$)--($(#1+#3,#2)$)--($(#1+#3,#2+#3)$)--($(#1,#2+#3)$)--cycle;
    \draw ($(#1+0.5*#3,#2+0.25*#3)$) node{{#5\color{#4}$\bullet$}};
    \draw ($(#1+0.25*#3,#2+0.5*#3)$) node{{#5\color{#4}$\bullet$}};
    \draw ($(#1+0.75*#3,#2+0.75*#3)$) node{{#5\color{#4}$\bullet$}};
    }
    
\newcommand{\SquareCircRed}[5]{
    \draw[line width=0.01mm]($(#1,#2)$)--($(#1+#3,#2)$)--($(#1+#3,#2+#3)$)--($(#1,#2+#3)$)--cycle;
    \draw ($(#1+0.25*#3,#2+0.5*#3)$) node{{#5\color{#4}$\bullet$}};
    }    
    
\newcommand{\SquareCircBlue}[5]{
    \draw[line width=0.01mm]($(#1,#2)$)--($(#1+#3,#2)$)--($(#1+#3,#2+#3)$)--($(#1,#2+#3)$)--cycle;
    \draw ($(#1+0.5*#3,#2+0.25*#3)$) node{{#5\color{#4}$\bullet$}};
    \draw ($(#1+0.75*#3,#2+0.75*#3)$) node{{#5\color{#4}$\bullet$}};
    }
    
\newcommand{\SquareCircBlueBis}[5]{
    \draw[line width=0.01mm]($(#1,#2)$)--($(#1+#3,#2)$)--($(#1+#3,#2+#3)$)--($(#1,#2+#3)$)--cycle;
    \draw ($(#1+0.75*#3,#2+0.75*#3)$) node{{#5\color{#4}$\bullet$}};
    }
    
\newcommand{\Circ}[4]{\draw ($(#1,#2)$) node{{#4\color{#3}$\bullet$}};}

\newcommand{\CrochetL}[3]{
\draw[line width=0.4mm](#1-0.2,#2)--(#1-0.2,#2+#3);
\draw[line width=0.4mm](#1-0.22,#2)--(#1-0.1,#2);
\draw[line width=0.4mm](#1-0.22,#2+#3)--(#1-0.1,#2+#3);
}

\newcommand{\CrochetR}[3]{
\draw[line width=0.4mm](#1+0.2,#2)--(#1+0.2,#2+#3);
\draw[line width=0.4mm](#1+0.22,#2)--(#1+0.1,#2);
\draw[line width=0.4mm](#1+0.22,#2+#3)--(#1+0.1,#2+#3);
}
\usetikzlibrary{fpu}
\usepackage{xint, xinttools}
\makeatletter
\def\@weierstrassgeneralterm #1#2#3{(0.1/#3*#2(#3*#1*pi r))}

\def\@weierstrassseries #1#2#3{
    \xintListWithSep{+}
                    {\xintApply {\@weierstrassgeneralterm{#1}{#2}}
                                {\xintApply{\xintiiPow {2}}{\xintSeq {0}{#3}}}}%
}

\def\SetWeierstrass #1#2{% #1=\x or \y, etc..., #2=summation from 0 to #2
    \fdef\weierstrasscos {\@weierstrassseries {#1}{cos}{#2}}%
    \fdef\weierstrasssin {\@weierstrassseries {#1}{sin}{#2}}%
}
\makeatother



\newcommand{\myacr}[1]{\acrlong*{#1} ou \acrshort*{#1}}
\let\mylistof\listof
\renewcommand\listof[2]{\mylistof{algorithm}{Liste des algorithmes}}
\renewcommand{\listfigurename}{Liste des figures}
% pour palier au problème de niveau des algos
\makeatletter
\providecommand*{\toclevel@algorithm}{0}
\makeatother

\usepackage{yhmath}

\usepackage{caption}
\usepackage{subcaption}

\usepackage{pstricks-add}

\usepackage{mathtools}



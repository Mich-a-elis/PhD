\chapter{Extension aux surfaces courbes}
\label{chap:surface_courbe}
\minitoc

\[\]

Dans ce chapitre, nous exposons une généralisation des notions abordés au deux chapitres précédents au cas des surfaces courbes dans l'espace. Plus précisement, on considère $\Omega$ comme une surface compacte orienté plongé dans $\mathbb{R}^3$. Pour tout point $p\in\Omega$ nous notons $N(p)$ la normale à la surface (autrement dit le vecteur normale à l'espace tangent $T_p\Omega$).

\section{Champ de croix, points singuliers, indices et séparatrices}

Dans cette section, nous revenons sur les définitions et concepts introduits dans le chapitre \ref{chap:theoritical}. Un champ de croix $\bar{u}$ défini sur $\Omega$ attribue à chaque point $p$ de $\Omega$ une croix (voir la définition \ref{def:croix}) dans l'espace tangent $T_p\Omega$. L'ensemble $\mathcal{S}_{\bar{u}_h}$, désignant les points singuliers de $\bar{u}_h$, est composé des points $p \in \Omega_h$ tels que $\bar{u}_h(p) = 0$. On dit que le champ de croix est presque-$\mathcal{C}^1$ s'il satisfait à la définition \ref{def:presqueC1}. En ajustant l'opérateur de rotation pour permettre la rotation d'une croix autour de la normale surfacique au point considéré, la proposition \ref{prop:cont1} autorise la rotation d'un champ de croix défini sur une surface par rapport à un champ d'angle défini sur cette même surface.\\\\
À la différence du cas planaire où une référence globale était disponible pour le calcul des angles des croix, dans ce contexte, nous choisissons une référence arbitraire pour chaque espace tangent de la surface, permettant ainsi de déterminer l'angle de la croix par rapport à cette référence. Cette approche conduit à la formule de l'indice d'un point $p$ dans un champ de croix presque-$\mathcal{C}^1$ défini sur $\Omega$ :
\begin{equation}
\label{eqn:index_surf}
id_{\bar{u}}(p) = \frac{1}{2\pi}\int_0^1 d\theta_{\bar{u}}^\gamma,
\end{equation}
où $\gamma$ est un chemin fermé inclus dans $\Omega$, paramétré dans le sens positif, encerclant uniquement le point $p$ sans inclure d'autres points singuliers de $\bar{u}$, et tel que la partie de $\Omega$ délimitée par $\gamma$ est homéomorphe à $\mathbb{R}^2$. Les définitions des lignes de champ pour les champs de croix sur les surfaces courbes ainsi que les propriétés des séparatrices restent applicables dans ce nouveau cadre. Cependant, il convient de noter que pour intégrer les lignes de champ à partir de points singuliers (séparatrices), le voisinage étoilé permettant de modifier localement le champ de croix pour faciliter le démarrage de la séparatrice doit être de taille suffisamment réduite. Plus précisément, ce voisinage doit être homéomorphe au plan pour mettre en œuvre l'opération de modification du champ introduite dans la proposition \ref{prop:stream_from_interior_sing}. À partir de là, les résultats établissant le lien entre le nombre de séparatrices associées à un point singulier et l'indice du point sont aisément retrouvés. Nous illustrons sur la figure \ref{fig:separatrice_illustration_surface} les séparatrices d'un point d'indice $1/4$ et d'un point d'indice $-1/4$.\\


\begin{figure}[!h]
  \centering
  \includegraphics[scale=0.282]{images/surface_sepa_5.pdf}
\\[0.5cm]
  \includegraphics[scale=0.282]{images/surface_sepa_3.pdf}
  \caption{Gauche: Illustration des séparatrices émanant de points singuliers d'indice $-1/4$ (en haut) et d'indice $1/4$ (en bas).}
  \label{fig:separatrice_illustration_surface}
\end{figure}

Examinons maintenant ce que devient ces concepts dans un cadre discret. Soit $\Omega_h$ un maillage triangulaire de $\Omega$ vérifiant les contraintes définis dans la sous-section \ref{subsec:mesh_tri} et soit $\bar{u}$ un champ de croix presque-$\mathcal{C}^1$ défini sur $\Omega$. L'évaluation de l'assertion \ref{ass:triangle_singulier} s'effectue en projetant les champ de croix sur un même espace tangent. On définit ensuite les zones singulières comme précedemment (voir \ref{subsec:discr_champ_de_croix}).


definition de delta avec cosinus\\
on definit ensuite les zone singuliere comme dans le chapitre 3\\
definir ensuite la represenatation en projetant sur les triangles\\
on se retrouve ainsi avec une definition par morceau

\ref{alg:discr_algo_main}


On peut subdiviser l'ensemble des triangles vérifiant l'assertion \ref{ass:triangle_singulier} en plusieurs partitions, dont les parties du plan correspondant seront appelées \emph{zones singulières}. Pour assembler ces partitions, on regroupe les triangles singuliers qui ont un ou plusieurs triangles singuliers comme voisins. Voici un exemple illustrant la construction de zones singulières sur la figure \ref{fig:zone_singuliere}. Dans ce cas, on se donne une configuration où la couleur rouge indique les lieux où l'une des propriétés de l'assertion \ref{ass:triangle_singulier} est vérifiée. Nous affichons ensuite les zones singulières correspondantes en bleu. Cette configuration particulière conduit à la création de cinq zones singulières que nous présentons sur la même figure. Dans la suite, nous utiliserons la notation $\mathbf{Z}=\cup_{i=1}^{N_Z}Z_i$ pour représenter la partie du plan occupé par l'ensemble des zones singulières. Ici, $Z_i$ désigne une zone singulière spécifique, et $N_z$ représente le nombre total de zones singulières. Ce nombre est nécessairefini puisque le champ de croix $\bar{u}$ possède un nombre fini de points singuliers. Nous associons ensuite à chaque zone singulière $Z\in \mathbf{Z}$ un point arbitrairement choisi dans $Z$, que nous notons $S_Z$ (par exemple, lorsque $Z$ est réduit à un unique triangle, on peut choisir $S_Z$ comme le barycentre du triangle en question).

\begin{figure}[h!]
\centering
\begin{subfigure}{0.49\textwidth}
    \includegraphics[width=\textwidth]{images/zone_singuliere_1.pdf}
    %\caption{Insertion de $D$.}
    %\label{fig:quad_eclatement}
\end{subfigure}
\hfill
\begin{subfigure}{0.49\textwidth}
    \includegraphics[width=\textwidth]{images/zone_singuliere_2.pdf}
    %\caption{Insertion de $E$.}
    %\label{fig:quad_carre}
\end{subfigure}
\caption{Illustration de la construction de zones singulières.}
\label{fig:zone_singuliere}
\end{figure}

Avec ces outils, nous sommes à présent en mesure de présenter notre proposition de représentation $\bar{u}_h$ sur $\Omega_h$ d'un champ de croix $\bar{u}$ défini sur $\Omega$. Pour tout $p\in\Omega_h$, nous définissons $\bar{u}_h$ de la manière suivante:\\

\begin{itemize}
\item[$\bullet$] si $p\in(\mathcal{S}_h\backslash\mathbf{Z})\cap\partial Z$, alors $\bar{u}_h(p)=\bar{u}(p)$.\\%[-0.3cm]
\item[$\bullet$] si $p\in\Omega_h\backslash\mathbf{Z}$ alors il existe $T\in\mathcal{T}_h$ tel que $p\in T$ et $T$ ne vérifie pas l'assertion \ref{ass:triangle_singulier}. On pose alors:
$$
\left\{
\begin{array}{l}
\theta_1 = \theta_{\bar{u}_h}(s_1)\\\\
\theta_2 = \theta_1 + \delta\theta(\bar{u}_h(s_1),\bar{u}_h(s_2))\\\\
\theta_3 = \theta_2 + \delta\theta(\bar{u}_h(s_2),\bar{u}_h(s_3))
\end{array}
\right.
$$
où $s_1$, $s_2$ et $s_3$ sont les sommets du triangle $T$. La croix $\bar{u}_h(p)$ est alors donnée par:
$$
\left\{
\begin{array}{l}
\bar{u}_h(p)=\displaystyle\left\{\mathbf{R}\left(\theta_p+m\frac{\pi}{2}\right)(1,0)^t,~m\in\mathbb{Z}\right\},\\\\
\theta_p=\displaystyle\sum_{i\in\llbracket1, 3\rrbracket}\lambda_i\theta_i,
\end{array}
\right.
$$
avec $(\lambda_i)_{i\in\llbracket 1, 3\rrbracket}$ les coordonnées barycentriques de $p$ dans le triangle $T$. Autrement dit, ils vérifient $p=\sum_{i=1}^3\lambda_i s_i$ et $\sum_{i=1}^3\lambda_i=1$.\\
\item[$\bullet$] si $p\in\mathbf{Z}$ alors il existe une zone singulière $Z\subset\mathbf{Z}$ tel que $p\in Z$. On a alors:\\
\begin{itemize}
 \item si $p=S_Z$, alors $\bar{u}_h(p)=0$.\\
 \item sinon la croix $\bar{u}_h(p)$ est donnée par:
\begin{equation}
\label{eqn:etoilage}
\left\{
\begin{array}{l}
\bar{u}_h(p)=\bar{u}_h(\widetilde{p}),\\\\
\{\widetilde{p}\}=[S_Zp)\cap\partial Z.
\end{array}
\right.
\end{equation}
Notons que l'ensemble $[S_Zp)\cap\partial Z$ est bien réduit à un singleton puisque $Z$ est une réunion de simplexes donc convexe.\\
\end{itemize}
\end{itemize}

\begin{remark}
La construction nous avons exposé se base sur l'angle signé entre les croix des arêtes du maillage. Plus précisément, nous utilisons les variations du champ de croix le long des arêtes pour construire notre représentation en chaque point et pour évaluer les indices des points singuliers. Ainsi, lorsque cette variation angulaire n'est pas définie, notre représentation induit des points singuliers non isolés, comme nous le verrons dans le lemme \ref{lem:isolation_pt_sing}.
Pour palier à ce désagrément, nous imposons dans toute la suite les conditions suivantes:\\
\begin{itemize}
 \item si on a $Z\subset\mathbf{Z}$ tel que $Z\cap\partial\Omega_h=\emptyset$, on doit avoir pour tout $p\in\partial Z\cap\mathcal{S}_h$, $\bar{u}(p)\neq 0$ et pour tout arête $s_1s_2\subset\partial Z$, $\delta\theta(\bar{u}(s_1),\bar{u}(s_2))$ doit être défini.\\
 \item si on a $Z\subset\mathbf{Z}$ tel que $Z\cap\partial\Omega_h\neq\emptyset$, on impose $S_Z\in\partial\Omega_h$ et on doit avoir pour tout $p\in(\partial Z\cap\mathcal{S}_h)\backslash\{S_Z\}$, $\bar{u}(p)\neq 0$ et pour tout arête $s_1s_2\subset\partial Z$, $\delta\theta(\bar{u}(s_1),\bar{u}(s_2))$ doit être défini.\\
\end{itemize}
Dans la pratique, il sera donc impératif de modifier un maillage qui ne satisfait pas ces contraintes. %La pertinence de ces contraintes prend tout son sens par la suite, avec l'approche de construction que nous proposons pour représenter $\bar{u}$ sur $\Omega_h$.
Étant donné que les points singuliers du champ de croix $\bar{u}$ sont isolés, on peut toujours satisfaire ces contraintes en affinant ou en modifiant localement (par exemple, par des opérations de retournement d'arêtes) un maillage qui ne satisfait pas ces contraintes.
\end{remark}





\section{Partitionnement de $\partial\Omega_h$ et maillage}
\label{sec:partitionnement_omega_h}

Nous abordons maintenant la question du partitionnement du maillage $\Omega_h$ représentant le domaine $\Omega_h$. De façon similare à ce que nous avons présenter dans le cas planaire, nous nous basons sur l'algorithme \ref{alg:discr_algo_main}.

Les points singuliers coincidant exactement avec les points $S_Z$ où $Z\subset\mathbf{Z}$ leur recherche consiste donc à retrouver ces points. Il faut ensuite déterminer le nombre de séparatrices par points singuliers ainsi que leur direction de départ grâce à la fonction $W_{p_0}^\gamma$ où ici $p_0$ désigne le point singulier et $\gamma$ est une paramétrisation du bord $\partial Z_{p_0}$ de la zone singuliere $Z_{p_0}$ contenant $p_0$. On intègre ensuite les lignes de champs sur la surface triangulée. Cependant, contrairement au cas planaire les triangles ne se trouve pas tous dans le même plan. Le passage d'une ligne de champ d'un triangle à son voisin se fait alors en projetant le vecteur d'intégration dans le plan du triangle voisin. Nous illustrons cela sur la figure ...


Une fois les points singuliers de $\bar{u}_h$ identifiés, nous procédons à la création des séparatrices sur $\Omega_h$.

\paragraph{Traversé d'un triangle singulier:}

\begin{figure}[!h]
\centering
\begin{subfigure}[b]{0.495\textwidth}
    \centering
    \includegraphics[width=\textwidth]{images/draw_streams_sing_1.pdf}
    \caption{\'Echec d'intégration d'une séparatrice dans un triangle singulier.}
    \label{fig:draw_streams_sing_1}
\end{subfigure}
\hfill
\begin{subfigure}[b]{0.495\textwidth}
    \centering
    \includegraphics[width=\textwidth]{images/draw_streams_sing_2.pdf}
    \caption{Processus d'intégration d'une séparatice dans un triangle singulier: étape 1.}
    \label{fig:draw_streams_sing_2}
\end{subfigure}
\\[0.8cm]
\begin{subfigure}[b]{0.495\textwidth}
    \centering
    \includegraphics[width=\textwidth]{images/draw_streams_sing_3.pdf}
    \caption{Processus d'intégration d'une séparatice dans un triangle singulier: étape 2.}
    \label{fig:draw_streams_sing_3}
\end{subfigure}
\hfill
\begin{subfigure}[b]{0.495\textwidth}
    \centering
    \includegraphics[width=\textwidth]{images/draw_streams_sing_4.pdf}
    \caption{Processus d'intégration d'une séparatice dans un triangle singulier: étape 3.}
    \label{fig:draw_streams_sing_4}
\end{subfigure}
\caption{Illustration de l'intégration d'une séparatice dans un triangle singulier.}
\label{fig:draw_streams_sing}
\end{figure}

\begin{figure}[!h]
\centering
\begin{subfigure}{0.65\textwidth}
    \includegraphics[width=\textwidth]{images/decoup_sans_fusion.pdf}
    \caption{Intégration de séparatrices sans fusion.}
    \label{fig:decoup_sans_fusion}
\end{subfigure}
\\[0.5cm]
\begin{subfigure}{0.65\textwidth}
    \includegraphics[width=\textwidth]{images/decoup_detect_fusion.pdf}
    \caption{Détection d'une fusion.}
    \label{fig:decoup_detect_fusion}
\end{subfigure}
\\[0.5cm]
\begin{subfigure}{0.65\textwidth}
    \includegraphics[width=\textwidth]{images/decoup_fusion.pdf}
    \caption{Intégration de séparatrices avec fusion.}
    \label{fig:decoup_fusion}
\end{subfigure}
\caption{Illustration de la fusion de séparatrices.}
\label{fig:fusion}
\end{figure}


Nous désignons par \emph{triangle singulier}, tout triangle contenant un point singulier. Lors de l'intégration d'une séparatrice, il peut arriver qu'elle doive traverser un triangle singulier. Dans ce cas, le processus d'intégration tel que décrit précédemment peut échouer (voir la figure \ref{fig:draw_streams_sing_1}). Cela résulte du fait que le champ d'angle associé au champ de croix dans le triangle n'est pas continue et présente des variations importantes. Pour surmonter ce problème, nous cherchons à représenter la séparatrice dans le triangle singulier en utilisant une succession d'autres segments calculés par un raffinement local du maillage à l'intérieur du triangle singulier. L'objectif est d'isoler le point singulier par rapport à la trajectoire de la séparatrice. Voici le procédé:

Considérons $T$ comme un triangle contenant un point singulier que doit traverser une séparatrice dont l'origine ne se trouve pas dans $T$. En d'autres termes, le dernier point calculé lors de la construction de cette séparatrice appartient à $T$ et le processus habituel d'intégration a généreé un segment traversant $T$ (voir figure \ref{fig:draw_streams_1}). Dans la suite nous appellerons ce "dernier point" le point d'entrée de la séparatrice dans $T$. On commence en subdivisant $T$ en quatre triangles. Cette subdivision peut être réalisée en reliant, par exemple, les milieux des arêtes de $T$ Il est essentiel de noter que si le point d'entrée coincide avec le milieu d'une arête, un autre point de cette arête doit être choisi pour la subdivision. L'objectif est que le point d'entrée se retrouve associé à un triangle issu de la subdivision et ne contenant pas de point singulier. On peut alors appliquer le processus d'intégration dans ce triangle, tel que décrit précédemment (voir figure \ref{fig:draw_streams_sing_2}). Il ne reste plus qu'à itérer ce schéma de construction jusqu'à ce que la séparatrice sorte de la partie du plan correspondant au triangle initial (voir figures \ref{fig:draw_streams_sing_3} et \ref{fig:draw_streams_sing_4}). On se retouve ainsi avec un raffinement local et adapté du maillage en fonction de la trajectoire de la séparatrice ce qui permet en pratique de ne pas raffiner l'intégralité du maillage pour avoir une bonne approximation de la séparatice. Il convient de souligner qu'il n'est pas nécessaire d'apporter des modifications locales au maillage dans sa globalité. Après avoir construit la séparatrice, nous conservons uniquement ses points de passage et supprimons tout raffinement local utilisé.


\paragraph{Fusion de séparatrices:}


de manière similaire à ce qui est réalisé dans \cite{marcon2019high}, les séparatrices du champ de croix sont construites simultanément en incrémentant chacune d'elles progressivement, et la rencontre entre deux séparatrices est anticipée en comparant à chaque incrément, d'une part, la distance entre les derniers points calculés et, d'autre part, les directions des derniers segments construits. En d'autres termes, on cherche à déterminer si, à un moment donné, deux séparatrices données avancent dans des directions opposées et si elles sont suffisamment proches l'une de l'autre. On compare ces deux mesures à des seuils prédéfinis, et en fonction du résultat, on décide de fusionner ou non les deux séparatrices.

La fusion se réalise en créant une nouvelle séparatrice par une fusion linéaire des points des deux séparatrices impliquées. Pour se faire, chaque séparatrice est prolongée à travers $\Omega_h$ jusqu'à atteindre la position de départ de l'autre, tout en maintenant le même nombre de points pour chacune des séparatrices. L'intérêt de fusionner les séparatrices réside dans la réduction de leur nombre, ce qui se traduit directement par une diminution du nombre de régions générées lors du découpage du domaine. Nous illustrons la fusion de deux séparatrices sur la figure \ref{fig:fusion}. Il est remarquable que la non-fusion des séparatrices génère davantage de régions, certaines étant fortement étirées et non homogènes par rapport aux autres, ce qui est une source potentielle d'inhomogénéité des mailles dans les maillages quadrilatéraux.


\subsection{Assemblage des partitions}

Une fois les séparatrices construites, nous procédons à la subdivision du maillage triangulaire $\Omega_h$ en plusieurs régions distinctes. Cette opération consiste à modifier $\Omega_h$ en y intégrant les segments formant les séparatrices, créant ainsi un nouveau maillage. Notre objectif est de délimiter chaque région sous forme de sous-maillage distinct de $\Omega_h$. Ce processus débute par l'identification des frontières de ces régions via la localisation des points d'intersection entre les séparatrices, suivie de l'extraction de chaque région individuelle.

\paragraph{Intersections de séparatrices:}

Comme annoncé, la première étape de ce processus consiste à déterminer les frontières des régions. La recherche des points d'intersection entre les séparatrices peut être réalisée localement dans chaque triangle et simultanément. Étant donné que les séparatrices sont constituées de segments, il suffit de vérifier l'intersection de ces segments les uns par rapport aux autres. En découpant les séparatrices au niveau de ces points d'intersection, cela met en évidence les contours des régions, comme illustré sur la figure \ref{fig:detect_intersection} avec une couleur différente pour chaque séparatice.

\paragraph{Identification des partitions:}
Dans cette phase, notre objectif est de découper le maillage triangulaire initial $\Omega_h$ en plusieurs sous-maillages, représentant chacun une partition. Pour y parvenir, nous enrichissons $\Omega_h$ en y insérant les points qui constituent les séparatrices, puis nous ajustons le maillage pour qu'i
\begin{figure}[h!]
\centering
\begin{subfigure}{0.49\textwidth}
    \includegraphics[width=\textwidth]{images/eclatement_1.pdf}
    %\caption{Insertion de $D$.}
    %\label{fig:quad_eclatement}
\end{subfigure}
\hfill
\begin{subfigure}{0.49\textwidth}
    \includegraphics[width=\textwidth]{images/intersect_stream.pdf}
    %\caption{Insertion de $E$.}
    %\label{fig:quad_carre}
\end{subfigure}
\caption{Intersection de séparatrices.}
\label{fig:detect_intersection}
\end{figure}


\begin{figure}[h!]
\centering
\begin{subfigure}{0.49\textwidth}
    \includegraphics[width=\textwidth]{images/eclatement_2.pdf}
    %\caption{Insertion de $D$.}
    %\label{fig:quad_eclatement}
\end{subfigure}
\hfill
\begin{subfigure}{0.49\textwidth}
    \includegraphics[width=\textwidth]{images/eclatement_3.pdf}
    %\caption{Insertion de $E$.}
    %\label{fig:quad_carre}
\end{subfigure}
\caption{Extraction des régions en tant que sous-maillages.}
\label{fig:eclatement}
\end{figure}


\section{Génération du maillage quadrilatéral}
\label{sec:gen_mesh_quad}

\subsection{Uniformisation}

\begin{figure}[h!]
\centering
\begin{subfigure}{0.49\textwidth}
    \includegraphics[width=\textwidth]{images/separatrice_echantillonage_1.pdf}
    %\caption{Insertion de $D$.}
    %\label{fig:quad_eclatement}
\end{subfigure}
\hfill
\begin{subfigure}{0.49\textwidth}
    \includegraphics[width=\textwidth]{images/separatrice_echantillonage_2.pdf}
    %\caption{Insertion de $E$.}
    %\label{fig:quad_carre}
\end{subfigure}
\caption{Exemple d'identification de séparatrices parallèles.}
\label{fig:separatrice_echantillonage}
\end{figure}


\begin{figure}[h!]
\centering
\begin{subfigure}{0.49\textwidth}
    \includegraphics[width=\textwidth]{images/quad_eclatement.pdf}
    %\caption{Insertion de $D$.}
    \label{fig:quad_eclatement}
\end{subfigure}
\hfill
\begin{subfigure}{0.49\textwidth}
    \includegraphics[width=\textwidth]{images/quad_carre.pdf}
    %\caption{Insertion de $E$.}
    \label{fig:quad_carre}
\end{subfigure}
\caption{Illustration du maillage des blocs résultant du partitionnement de l'exemple présenté à la figure \ref{fig:separatrice_echantillonage}.}
\label{fig:maillage_quad_carre}
\end{figure}

\color{black}

\section{Opération d'alignement}

Abordons à présent la discrétisation de l'opération d'alignement exposée dans le chapitre \ref{chap:theoritical}, en supposant que $\Omega$ soit un domaine simplement connexe. Pour rappel, cette opération vise à créer un champ de croix $\bar{v}=\mathbf{R}(\phi)\bar{u}$ aligné sur $\partial\Omega$ (c'est-à-dire aligné sur le champ de croix $\bar{n}$ associé à la normale sortante de $\partial\Omega$). Ceci est réalisé à partir d'un champ de croix $\bar{u}$ presque-$\mathcal{C}^1$ défini sur $\Omega$, mais non nécessairement aligné par rapport au bord de $\Omega$. La fonction $\phi$ est définie par l'équation \eqref{eqn:principe_def_phi}.

Considérons $\bar{u}_h$ comme une représentation discrète de $\bar{u}$ sur $\Omega_h$. Dans le contexte discret, cette opération équivaut à aligner $\bar{u}_h$ sur un champ de croix défini sur $\partial\Omega_h$ et représentant le champ de croix associé à la normale extérieure de $\partial\Omega$. Commençons par construire cette représentation que nous notons $\bar{n}_h$. Pour tout sommet $a\in\mathcal{S}_h\cap\partial\Omega_h$, avec $b$ et $c$ comme ses sommets voisins également situés sur $\partial\Omega_h$, nous avons :
\begin{equation}
\bar{n}_h(a)=\displaystyle\left\{\mathbf{R}\left(\frac{m\pi}{2}\right)g(a),~ m\in\mathbb{Z}\right\},
\end{equation}
où $g(a)$ est donné par:
\begin{equation}
g(a)=\displaystyle\frac{1}{\|\overrightarrow{ab}\|+\|\overrightarrow{ac}\|}\left(\overrightarrow{ab}+\overrightarrow{ac}\right).
\end{equation}
Ensuite, on élimine les arêtes de bord où la variation angulaire $\delta\theta$ du champ $\bar{n}_h$ n'est pas définie, soit en effectuant un raffinement local ou global du maillage, soit en insérant dans le maillage un nouveau point situé au milieu de l'arête en question. Ce point sera considéré comme un point singulier dans le champ $\bar{n}_h$, et par conséquent, pour un tel point $a\in\mathcal{S}_h\cap\partial\Omega_h$, nous fixons $\bar{n}_h(a)=0$. Ainsi, on se retrouve avec deux types d'arêtes incluses dans $\partial\Omega_h$ : les arêtes avec un sommet singulier dans $\bar{n}_h$ et les arêtes où la variation angulaire $\delta\theta$ de $\bar{n}_h$ est bien définie. La construction de $\bar{n}_h$ se poursuit alors de la manière suivante : pour tout point $p\in ab$ avec $ab$ une arête incluse dans $\partial\Omega_h$, on a:\\
\begin{itemize}
\item[$\bullet$] Si $\delta\theta(\bar{n}_h(a), \bar{n}_h(b))$ est défini, alors:
 $$
 \left\{
 \begin{array}{l}
 \bar{n}_h(p)=\mathbf{R}(\theta_p)\bar{n}_h(a),\\\\
 \theta_p = \theta_{\bar{n}_h}(a)+\displaystyle\frac{\|\overrightarrow{ap}\|}{\|\overrightarrow{ab}\|}\delta\theta(\bar{n}_h(a), \bar{n}_h(b)).
 \end{array}
 \right.
 $$
 \item[$\bullet$] Sinon, par construction, l'un des deux sommets de l'arête $ab$ est nul et l'autre non nul. Supposons sans perte de généralité qu'il s'agit de $a$. On pose alors:
 $$
 \bar{n}_h(p)=\bar{n}_h(b).
 $$
\end{itemize}
Après la construction du champ de croix $\bar{n}_h$, nous entamons l'opération d'alignement proprement dite. L'objectif est d'ajuster $\bar{u}_h$ pour former un nouveau champ de croix $\bar{v}_h$ tel que $\bar{v}_h=\bar{n}_h$ sur $\Omega_h$, à l'exception d'un nombre limité de points où il s'annule. En d'autres termes, nous visons à ce que, pour tout $p\in\partial\Omega_h$, $\bar{v}_h(p)\in\{\bar{n}_h(p), 0\}$. Pour y parvenir, nous débutons en associant à chaque sommet $p\in\partial\mathcal{S}_h\cap\Omega_h$ un paramètre $I_p$ vérifiant:
\begin{equation}
I_p=\displaystyle\frac{k}{4}\mbox{ avec }k\in\mathbb{Z}\mbox{ et }k\leq 1
\end{equation}
Ce paramètre représente l'indice désiré pour les points de bord dans le champ de croix $\bar{v}_h$ (une fois l'opération d'alignement réalisée). Un critère a priori pour choisir automatiquement ce paramètre a été exposé dans la sous-section \ref{subsec:sing_bord}. Ensuite, nous définissons l'ensemble $\mathcal{B}$ qui regroupe les points que nous souhaitons rendre singuliers dans le champ de croix $\bar{v}_h$ (une fois l'opération d'alignement réalisée). Cet ensemble doit, naturellement, inclure tous les points $p$ tels que $I_p\neq 0$, car ces points auront un indice non nul dans $\bar{v}_h$. Cependant, il est également envisageable d'inclure dans $\mathcal{B}$ des points dont le paramètre $I_p$ est nul. L'ensemble $\mathcal{B}$ doit aussi inclure les points singuliers du champ $\bar{n}_h$. Ces points étant déjà singulier dans $\bar{n}_h$, ils resteront singuliers dans $\bar{v}_h$ après l'opération d'alignement. Pour finir, il est impératif que le champ de croix $\bar{u}_h$ satisfasse:\\
\begin{itemize}
 \item[$\bullet$] $0\leq Card(\mathcal{S}_{\bar{u}_h})<\infty$,\\
 \item[$\bullet$] Pour tout point $p\in\mathcal{S}_{\bar{u}_h}$, $id_{\bar{u}_h}(p)=k/4$, avec $k\in\mathbb{Z}$ et $k\leq 1$,\\
 \item[$\bullet$] Soit $\gamma$ une paramétrisation de $\partial\Omega_h$ dans le sens positif. Il existe $\theta_{\bar{u}_h}^\gamma$ vérifiant:
 \begin{equation}
    \label{eqn:etude_hyp_u_simple}
    \theta_{\bar{u}_h}^\gamma(1)-\theta_{\bar{u}_h}^\gamma(0)=\chi(\Omega_h)-\sum_{p\in\mathcal{B}}I_p.
\end{equation}
\end{itemize}
Ces critères sont automatiquement vérifiés si les points singuliers de $\bar{u}$ sont isolés et que le maillage $\Omega_h$ est adapté à la représentation discrète que nous avons présentée dans la sous-section \ref{sec:repr_discrete}. Le champ de croix $\bar{v}_h$ est alors défini pour tout $p\in\Omega_h$ par:
\begin{equation}
\bar{v}_h(p)=
\left\{
\begin{array}{ll}
\mathbf{R}(\phi_h(p))\bar{u}_h(p) & \mbox{ si } p\in\Omega_h\backslash(\mathcal{B}\cup\mathcal{S}_{\bar{u}_h}),\\[0.5cm]
\bar{n}_h(p) & \mbox{ si } p\in(\mathcal{S}_{\bar{u}_h}\cap\partial\Omega_h)\backslash\mathcal{B},\\[0.5cm]
0 & \mbox{ si } p\in\mathcal{B}.
\end{array}
\right.
\label{eqn:etude_def_v_second}
\end{equation}
où $\phi_h$ est définie par l'équation de Laplace suivant:
\begin{equation}
\left\{
\begin{array}{lcll}
\Delta\phi_h &=& 0 &\mbox{ dans }\Omega,\\[0.5cm]
\phi_h(\gamma(t))&=&\theta_{\bar{n}_h}^\gamma(t)+\mathcal{I}(t)-\theta_{\bar{u}_h}(\gamma(t))& \mbox{ sur } \gamma^{-1}(\partial\Omega_h\backslash(\mathcal{B}\cup\mathcal{S}_{\bar{u}_h})),
\end{array}
\right.
\label{eqn:algorithm_def_phi}
\end{equation}
où la fonction $\mathcal{I}$ est donnée par:
$$
\mathcal{I}(t)=\displaystyle\sum_{s\in\gamma^{-1}(\mathcal{B})}\left[\left(\pi-\widehat{\gamma(s)}-2\pi I_{\gamma(s)}\right)-\left(\displaystyle\lim\limits_{r\rightarrow s^+}\theta^{\gamma}_{\bar{n}_h}(r) - \lim\limits_{r\rightarrow s^-}\theta^{\gamma}_{\bar{n}_h}(r)\right)\right]\mathbb{1}_{[0, t]}(s),
$$
avec $\widehat{\gamma(s)}$ la mesure de l'ouverture angulaire de la frontière en $\gamma(s)$.

En pratique, la résolution de l'équation \eqref{eqn:algorithm_def_phi} est effectuée par la méthode des éléments finis $\mathbb{P}_1$, en utilisant la formule du Laplacien cotangente pour la discrétisation de l'opérateur Laplacien \cite{solomon2014laplace, nealen2006laplacian, belkin2008discrete} (voir Annexe \ref{chap:annexes} pour plus de détails). Cette approche est privilégiée en raison de sa simplicité, facilitant l'extension de l'opération d'alignement dans le cas des surfaces courbes dans l'espace (voir le chapitre \ref{chap:surface_courbe}).

\begin{remark}
\[\]
\vspace{-1cm}
\begin{enumerate}
\item Le champ de croix $\bar{v}_h$ conserve certaines propriétés du champ de croix $\bar{u}_h$. Ainsi, on a $\mathcal{S}_{\bar{v}_h}\cap(\Omega_h\backslash\partial\Omega_h)=\mathcal{S}_{\bar{u}_h}\cap(\Omega_h\backslash\partial\Omega_h)$, puisque $\phi_h$ est $\mathcal{C}^0$ sur $\Omega_h\backslash\partial\Omega_h$. Les indices des points constituant ces ensembles sont également préservés. Autrement dit, pour tout point $p\in\Omega_h\backslash\partial\Omega_h$, on a $id_{\bar{u}_h}(p)=id_{\bar{v}_h}(p)$.
\item Pour tout $p\in\partial\Omega_h$, si $p\in\mathcal{S}_h$, alors $id_{\bar{v}_h}(p)=I_p$ ; sinon, $id_{\bar{v}_h}
(p)=0$.
\item Lorsque $\Omega_h$ est un domaine non simplement connexe, la procédure d'alignement reste conforme à celle développée dans le chapitre \ref{chap:theoritical}. Il convient toutefois de définir avec précision $\bar{n}_h$ et $\mathcal{B}$ pour chaque composante connexe de $\partial\Omega_h$.
\end{enumerate}
\end{remark}
Nous présentons le processus d'alignement à travers un exemple illustré à la figure \ref{fig:alignment}.

\begin{figure}[h!]
\centering
\begin{subfigure}{0.515\textwidth}
    \includegraphics[width=\textwidth]{images/alignment_1.pdf}
    \caption{Champ de croix $\bar{u}_h$}
    \label{fig:alignment_1}
\end{subfigure}
\hfill
\begin{subfigure}{0.478\textwidth}
    \includegraphics[width=\textwidth]{images/alignment_2.pdf}
    \caption{Champ scalaire $\phi_h$ (en randian)}
    \label{fig:alignment_2}
\end{subfigure}
\\[0.5cm]
\begin{subfigure}{0.51\textwidth}
    \includegraphics[width=\textwidth]{images/alignment_3.pdf}
    \caption{Champ de croix $\bar{v}_h$ (suite à l'opération d'alignement).}
    \label{fig:alignment_3}
\end{subfigure}
\hfill
\begin{subfigure}{0.482\textwidth}
    \includegraphics[width=\textwidth]{images/alignment_4.pdf}
    \caption{Maillage.}
    \label{fig:alignment_4}
\end{subfigure}
\caption{Illustration du processus d'alignement dans un cadre discret.}
\label{fig:alignment}
\end{figure}





















































\section{Extension des résultats principaux du cas planaire}

somme de champs de croix\\
heat method\\
connexion\\

\subsection{Operation d'alignement}


\subsection{Etude de la méthode}
\label{subsec:etude_de_la_methode}
Dans la section précédente, nous avons exposé des algorithmes qui, sous certaines conditions, permettent de subdiviser un domaine en régions à quatres côtés. Dans cette partie, nous allons présenter plusieurs résultats visant à garantir l'efficacité de ces méthodes. Commençons par le cas où le champ de croix est aligné avec le bord du domaine. Nous avons le résultat suivant:
\begin{theorem}
\label{thm:theorem1}
Soit $\Omega$ un domaine borné et fermé dans $\mathbb{R}^2$ avec une frontière régulière par morceau et soit $\bar{u}$ un champ de croix presque-$\mathcal{C}^1$ aligné avec $\partial\Omega$ tel que $0<Card(\mathcal{S}_{\bar{u}})<\infty$ et pour tout $p\in\Omega$, $id_{\bar{u}}(p)=k/4$ où $k\in\mathbb{Z}$ et $k\leq 1$. Si l'algorithme de partitionnement \ref{alg:algo_main} appliqué à $\bar{u}$ converge alors le partitionnement résultant est une décomposition de $\Omega$ en régions à quatre côtés.
\end{theorem}

\begin{proof}
Selon les propositions \ref{prop:stream_from_interior_sing} et \ref{prop:stream_from_bord_sing}, chaque point singulier de $u$ donne lieu à un nombre fini de séparatrices. Étant donné que ces séparatrices ne convergent pas vers des cycles limites, elles doivent soit se terminer en un point singulier, soit intersecter la frontière de $\Omega$. En conséquence, les séparatrices de $\bar{u}$ divisent $\Omega$ en régions qui ne contiennent aucun point singulier et qui sont délimitées par des séparatrices.

Soit $\mathcal{R}$ l'une de ces régions. Selon le Théorème de Poincaré-Hopf, on a $\chi(\mathcal{R})=\sum_{i=1}^{n_c} id_{\bar{u}}(c_i)$, où $(c_i)_{i\in\llbracket1,n_c\rrbracket}$ désigne les coins de $\mathcal{R}$ (c'est à dire les points d'intersections des séparatrices formant la région $\mathcal{R}$) et $n_c$ est le nombre de coins. $\chi(\mathcal{R})\geq0$ car, d'après la proposition \ref{prop:loveprop}, nous avons $id_{\bar{u}}(c_i)=1/4>0$ pour chaque coin $c_i$ de $\mathcal{R}$. De plus, on sait que $\chi(\mathcal{R}) = 2 - 2g - b$, où $g$ représente le genre de $\mathcal{R}$ et $b$ le nombre de composantes connexes de $\partial\mathcal{R}$. Comme $\mathcal{R}$ est une partie du plan, on a $g = 0$, ce qui implique que $\chi(\mathcal{R})\in\{0, 1\}$.
\begin{itemize}
\item $\chi(\mathcal{R})=0$: dans ce cas on a $n_c=0$ et $\mathcal{R}=\Omega$. Cela mène à une contradiction puisque, selon les hypothèses du théorème on a $Card(\mathcal{S}_{\bar{u}})>0$.
\item $\chi(\mathcal{R})=1$: cela implique que :
$$1=\sum_{i=1}^{n_c}id_u(c_i)=\sum_{i=1}^{n_c}\frac{1}{4}\Longrightarrow n_c=4.$$
Ainsi, $\mathcal{R}$ est une région à quatre côtés.
\end{itemize}
\end{proof}

\paragraph{\'Etude de l'opération d'alignement}

Soit $\Omega$ un domaine fermé et borné dans $\mathbb{R}^2$ avec une frontière lisse par morceaux. Supposons dans un premier temps que $\Omega$ est simplement connexe. Soit $\mathcal{B}\subset\partial\Omega$ un ensemble de point isolé de $\partial\Omega$ et $I_p$ un paramètre associé à chaque point $p\in\Omega$ tel que:
\begin{equation}
I_p=
\left\{
\begin{array}{ll}
\displaystyle\frac{k}{4}\mbox{ avec }k\in\mathbb{Z}\mbox{ et }k\leq 1& \mbox{ si } p\in\mathcal{B}\\[0.5cm]
0& \mbox{ sinon }
\end{array}
\right.
\label{eqn:etude_def_I}
\end{equation}
Soit $\bar{u}$ un champ de croix presque-$\mathcal{C}^1$ défini sur $\Omega$, non nécessairement aligné sur $\partial\Omega$, tel que $0<Card(\mathcal{S}_{\bar{u}})<\infty$ et pour tout point $p\in\mathcal{S}_{\bar{u}}\backslash\partial\Omega$, $id_{\bar{u}}(p)=k/4$, avec $k\in\mathbb{Z}$ et $k\leq 1$. On suppose de plus qu'il existe $\theta_{\bar{u}}^\gamma$ un relèvement continu de $\bar{u}$ le long de $\gamma$ tel que:
\begin{equation}
    \label{eqn:etude_hypothese_u}
    \theta_{\bar{u}}^\gamma(1)-\theta_{\bar{u}}^\gamma(0)=2\pi\chi(\Omega)-2\pi\sum_{p\in\mathcal{B}}I_p.
\end{equation}
où $\gamma$ est une paramétrisation sur $[0, 1]$ de $\partial\Omega$ orientée positivement et vérifiant $\gamma(0)=\gamma(1)\notin\mathcal{B}\cup\mathcal{S}_{\bar{n}}\cup\mathcal{S}_{\bar{u}}$.
Soit $\phi$, la fonction définie par l'équation de Laplace suivant:
\begin{equation}
\left\{
\begin{array}{lcll}
\Delta\phi &=& 0 &\mbox{ dans }\Omega,\\[0.5cm]
\phi(\gamma(t))&=&\theta_{\bar{n}}^\gamma(t)+\mathcal{I}(t)-\theta_{\bar{u}}(\gamma(t))& \mbox{ sur } \gamma^{-1}(\partial\Omega\backslash(\mathcal{B}\cup\mathcal{S}_{\bar{n}}\cup\mathcal{S}_{\bar{u}})),
\end{array}
\right.
\label{eqn:etude_def_phi}
\end{equation}
où la fonction $\mathcal{I}$ est donnée pour tout $t\in\gamma^{-1}(\partial\Omega\backslash(\mathcal{B}\cup\mathcal{S}_{\bar{n}}\cup\mathcal{S}_{\bar{u}}))$ par:
$$
\mathcal{I}(t)=\sum_{s\in\gamma^{-1}(\mathcal{B}\cup\mathcal{S}_{\bar{n}})}\left[\left(\pi-\widehat{\gamma(s)}-2\pi I_{\gamma(s)}\right)-\left(\lim\limits_{r\rightarrow s^+}\theta^{\gamma}_{\bar{n}}(r) - \lim\limits_{r\rightarrow s^-}\theta^{\gamma}_{\bar{n}}(r)\right)\right]\mathbb{1}_{[0, t]}(s),
$$
avec $\widehat{\gamma(s)}$ la mesure de l'ouverture angulaire de la frontière en $\gamma(s)$. Nous avons alors le lemme suivant:

\begin{lemma}
    \label{lem:marvelous_lemma}
    $\phi(\gamma(1))-\phi(\gamma(0))=0$.
\end{lemma}
\begin{proof}
    $$
    \begin{array}{lcl}
        \phi(\gamma(1))-\phi(\gamma(0))&=&\theta^\gamma_{\bar{n}}(1)+\mathcal{I}(1)-\theta_{\bar{u}}(\gamma(1))-(\theta^\gamma_{\bar{n}}(0)+\mathcal{I}(0)-\theta_{\bar{u}}(\gamma(0)))\\\\
        &=&(\theta^\gamma_{\bar{n}}(1)-\theta_{\bar{n}}^\gamma(0))+(\mathcal{I}(1)-\mathcal{I}(0))-(\theta_{\bar{u}}(\gamma(1))-\theta_{\bar{u}}(\gamma(0))
    \end{array}
    $$
    En posant $\gamma^{-1}(\mathcal{B}\cup\mathcal{S}_{\bar{n}})=\{t_1,\dots,t_{n_t}\}$ avec $n_t\in\mathbb{N}^*$ et $t_1<\dots<t_{n_t}$, on a:
    $$
    \begin{array}{lcl}
        \theta_{\bar{n}}^\gamma(1)-\theta^\gamma_{\bar{n}}(0)&=&\theta_{\bar{n}}^\gamma(t_1^-)-\theta_{\bar{n}}^\gamma(0)+\displaystyle\sum_{i=2}^{n_t}\left(\theta_{\bar{n}}^\gamma(t_i^-)-\theta_{\bar{n}}^\gamma(t_{i-1}^+)\right)+\theta_{\bar{n}}^\gamma(0)\\[0.7cm]
        &&+\theta_{\bar{n}}^\gamma(1)-\theta_{\bar{n}}^\gamma(t_{n_t}^-)+\displaystyle\sum_{i=1}^{n_t}\left(\theta_{\bar{n}}^\gamma(t_i^+)-\theta_{\bar{n}}^\gamma(t_i^-)\right)-\theta_{\bar{n}}^\gamma(0)\\[0.7cm]
        &=&\displaystyle\int_0^1d\theta_{\bar{n}}^\gamma+\displaystyle\sum_{i=1}^{n_t}\left(\theta_{\bar{n}}^\gamma(t_i^+)-\theta_{\bar{n}}^\gamma(t_i^-)\right)\\[0.7cm]
        \theta_{\bar{n}}^\gamma(1)-\theta^\gamma_{\bar{n}}(0)&=&\displaystyle\int_0^1d\theta_{\bar{n}}^\gamma+\displaystyle\sum_{s\in\gamma^{-1}(\mathcal{B}\cup\mathcal{S}_{\bar{n}})}\left(\theta_{\bar{n}}^\gamma(s^+)-\theta_{\bar{n}}^\gamma(s^-)\right)
    \end{array}
    $$
    Remarquons aussi que puisque $\mathcal{I}(0)=0$, on a:
    $$
    \mathcal{I}(1)-\mathcal{I}(0)=\sum_{s\in\gamma^{-1}(\mathcal{B}\cup\mathcal{S}_{\bar{n}})}\left[\left(\pi-\widehat{\gamma(s)}-2\pi I_{\gamma(s)}\right)-\left(\theta^{\gamma}_{\bar{n}}(s^+)-\theta^{\gamma}_{\bar{n}}(s^-)\right)\right].
    $$
    Il vient alors que:
    $$
    \phi(\gamma(1))-\phi(\gamma(0))=\displaystyle\int_0^1d\theta_{\bar{n}}^\gamma+\sum_{s\in\gamma^{-1}(\mathcal{B}\cup\mathcal{S}_{\bar{n}})}\left(\pi-\widehat{\gamma(s)}-2\pi I_{\gamma(s)}\right)-(\theta_{\bar{u}}^\gamma(1)-\theta_{\bar{u}}^\gamma(0)).
    $$
    Autrement dit,
    $$
    \begin{array}{lcl}
        \phi(\gamma(1))-\phi(\gamma(0))&=&\displaystyle\sum_{i=1}^{n_t}\int_{t_i}^{t_{i+1}}d\theta_{\bar{n}}^\gamma+\sum_{s\in\gamma^{-1}(\mathcal{B})}\left(\pi-\widehat{\gamma(s)}\right)\\[0.7cm]
        &&-\displaystyle\left(\sum_{s\in\gamma^{-1}(\mathcal{B}\cup\mathcal{S}_{\bar{n}})}2\pi I_{\gamma(s)}+(\theta_{\bar{u}}^\gamma(1)-\theta_{\bar{u}}^\gamma(0))\right)
    \end{array}
    $$
    où on a posé $t_{n_t+1}:=t_1$. D'après le théorème des tangentes tournantes \cite{hopf1935drehung, rotskoff2010gauss}, on sait que :
    $$
    \displaystyle\sum_{i=1}^{n_t}\left(\theta_{\bar{n}}^\gamma(t_{i+1})-\theta_{\bar{n}}^\gamma(t_i)\right)+\sum_{s\in\gamma^{-1}(\mathcal{B}\cup\mathcal{S}_{\bar{n}})}\left(\pi-\widehat{\gamma(s)}\right)=2\pi.
    $$
    On obtient alors en utilisant la condition \eqref{eqn:etude_hypothese_u}
    $$
    \phi(\gamma(1))-\phi(\gamma(0))=2\pi-2\pi\chi(\Omega).
    $$
    Autrement dit,
    $$
    \phi(\gamma(1))-\phi(\gamma(0))=0.
    $$
\end{proof}
L'opération d'alignement consiste à calculer le champ de croix $\bar{v}$ défini pour tout $p\in\Omega$ par:
\begin{equation}
\bar{v}(p)=
\left\{
\begin{array}{ll}
\mathbf{R}(\phi(p))\bar{u}(p) & \mbox{ si } p\in\Omega\backslash(\mathcal{B}\cup\mathcal{S}_{\bar{n}}\cup\mathcal{S}_{\bar{u}}),\\[0.5cm]
\bar{n}(p) & \mbox{ si } p\in(\mathcal{S}_{\bar{u}}\cap\partial\Omega)\backslash(\mathcal{B}\cup\mathcal{S}_{\bar{n}}),\\[0.5cm]
0 & \mbox{ si } p\in\mathcal{B}\cup\mathcal{S}_{\bar{n}}.
\end{array}
\right.
\label{eqn:etude_def_v}
\end{equation}
Nous avons alors le théorème suivant:
\begin{theorem}
\label{thm:theorem2}
Le champ de croix $\bar{v}$ est presque-$\mathcal{C}^1$ sur $\Omega$ et aligné avec $\partial\Omega$. De plus, pour tout $p\in\Omega$, on a:
\begin{equation}
id_{\bar{v}}(p)=
\left\{
\begin{array}{ll}
    id_{\bar{u}}(p) & \mbox{ si } p\in\Omega\backslash\partial\Omega,\\\\
    I_p & \mbox{ sinon }.
\end{array}
\right.
\end{equation}
\end{theorem}

\begin{proof}
    Du fait que la fonction $\phi$ définie par l'équation \eqref{eqn:etude_def_phi} est de classe $\mathcal{C}^1$ sur $\Omega\backslash(\mathcal{B}\cup\mathcal{S}_{\bar{n}}\cup(\mathcal{S}_{\bar{u}}\cap\partial\Omega)))$, le champ $\bar{v}$ est presque-$\mathcal{C}^1$ sur $\Omega$. Cela découle immédiatement de la proposition \ref{prop:cont1}, étant donné que $\bar{u}$ est presque-$\mathcal{C}^1$ sur $\Omega$.

    De plus, $\bar{v}$ est aligné avec $\partial\Omega$. En effet, $\bar{v}(p)=0$ pour tout $p\in\mathcal{B}\cup\mathcal{S}_{\bar{n}}$ et $\bar{v}(p)=\bar{n}(p)$ pour tout $p\in(\mathcal{S}_{\bar{u}}\cap\partial\Omega)\backslash(\mathcal{B}\cup\mathcal{S}_{\bar{n}})$. Par ailleurs, pour tout $p\in\partial\Omega\backslash(\mathcal{B}\cup\mathcal{S}_{\bar{n}}\cup\mathcal{S}_{\bar{u}})$ on a:
    $$\theta_{\bar{v}}(\gamma(t_p))=\phi(\gamma(t_p))+\theta_{\bar{u}}^\gamma(t_p)+k\frac{\pi}{2}=\theta_{\bar{n}}^\gamma(t_p)+\mathcal{I}(t_p)-\theta_{\bar{u}}^\gamma(t_p)+\theta_{\bar{u}}^\gamma(t_p)+k\frac{\pi}{2},$$
    où $t_p\in[0,1]$ tel que $\gamma(t_p)=p$ et $k\in\mathbb{Z}$. Or pour tout $s\in\gamma^{-1}(\mathcal{B}\cup\mathcal{S}_{\bar{n}})\cap[0, t_p]$ on a:
    $$
    \lim\limits_{r\rightarrow s^+}\theta^{\gamma}_{\bar{n}}(r) - \lim\limits_{r\rightarrow s^-}\theta^{\gamma}_{\bar{n}}(r)\equiv(\pi-\widehat{\gamma(s)})\pmod{\frac{\pi}{2}}.
    $$
    Il vient alors que:
    $$
    \theta_{\bar{v}}(\gamma(t_p))=\theta_{\bar{n}}^{\gamma}(t_p)+\displaystyle\sum_{s\in\gamma^{-1}(\mathcal{B}\cup\mathcal{S}_{\bar{n}})}\left[\left(\pi-\widehat{\gamma(s)}-m_s\frac{\pi}{2}\right)-\left(\pi-\widehat{\gamma(s)}+k_s\frac{\pi}{2}\right)\right]\mathbb{1}_{[0, t_p]}(s)+k\frac{\pi}{2},
    $$
    où $k_s\in\mathbb{Z}$ et où on a posé $I_{\gamma(s)}=m_s/4$ avec $m_s\in\mathbb{Z}$ et $m_s\leq1$. Autrement dit, on a:
    $$
    \theta_{\bar{v}}(\gamma(t_p))=\theta_{\bar{n}}^\gamma(t_p)+\displaystyle\sum_{s\in\gamma^{-1}(\mathcal{B}\cup\mathcal{S}_{\bar{n}})}\left[-m_s\frac{\pi}{2}+k_s\frac{\pi}{2}\right]\mathbb{1}_{[0, t_p]}(s)+k\frac{\pi}{2}$$
    et donc:
    $$
    \theta_{\bar{v}}(\gamma(t_p))\equiv\theta_{\bar{n}}^\gamma(t_p)\pmod{\frac{\pi}{2}}.
    $$
    Ce qui implique que $\bar{v}(p)=\bar{n}(p)$. On conclut alors que $\bar{v}$ est aligné avec $\partial\Omega$.\\\\
    Nous calculons maintenant pour tout $p\in\Omega$, l'indice de $p$ dans le champ $\bar{v}$. Ainsi, on a:\\
    \begin{itemize}
        \item[$\bullet$] Si $p\in\Omega\backslash\partial\Omega$, $id_{\bar{v}}(p)=id_{\bar{u}}(p)$ selon la proposition \ref{prop:relation_u_Rthetau}, car la fonction $\phi$ définie par l'équation \eqref{eqn:etude_def_phi} est de classe $\mathcal{C}^1$ sur $\Omega\backslash\partial\Omega$.\\
        \item[$\bullet$] Si $p\in\partial\Omega\backslash\{\gamma(0)\}$ (sachant que $\gamma(0)=\gamma(1)$), on a:
        $$
        id^\partial_{\bar{v}}(p)=\frac{1}{2\pi}\left[\pi-\widehat{p}+\displaystyle\lim\limits_{s\rightarrow 0}\int_s^{1-s}d\theta_{\bar{v}}^{\mathcal{C}}\right]=\frac{1}{2\pi}\left(\pi-\widehat{p}+(\theta_{\bar{v}}^{\mathcal{C}}(1)-\theta_{\bar{v}}^{\mathcal{C}}(0))\right),
        $$
        où $\mathcal{C}$ est un lacet paramétré sur $[0, 1]$ tel que $\mathcal{C}(0)=p=\mathcal{C}(1)$ et les vecteurs $\mathcal{C}'(0)$ et $\mathcal{C}'(1)$ sont tangents à $\partial\Omega$. De plus, $\mathcal{C}$ n'englobe aucun autre point singulier de $\bar{u}$. Soit $t_p\in]0, 1[$ tel que $\gamma(t_p)=p$. On a:
        $$
        id^\partial_{\bar{v}}(p)=\frac{1}{2\pi}\left[\pi-\widehat{p}+\left(\theta_{\bar{v}}^\gamma(t_p^-)-\theta_{\bar{v}}^\gamma(t_p^+)\right)\right]
        $$
        où on a noté $\lim_{t\rightarrow t_p^+}\theta^\gamma_{\bar{v}}(t)=\theta^\gamma_{\bar{v}}(t_p^+)$ et $\lim_{t\rightarrow t_p^-}\theta^\gamma_{\bar{v}}(t)=\theta_{\bar{v}}^\gamma(t_p^-)$. Autrement dit, on a:
        $$
        \begin{array}{lcl}
        id^\partial_{\bar{v}}(p)&=&\displaystyle\frac{1}{2\pi}\left[\pi-\widehat{p}+\left(\phi(\gamma(t_p^-))+\theta_{\bar{u}}^\gamma(t_p^-)-\phi(\gamma(t_p^+))-\theta_{\bar{u}}^\gamma(t_p^+)\right)\right]\\\\
        &=&\displaystyle\frac{1}{2\pi}\left[\pi-\widehat{p}+\left(\theta_{\bar{n}}^\gamma(t_p^-)-\theta_{\bar{n}}^\gamma(t_p^+)\right)+\left(\mathcal{I}(t_p^-)-\mathcal{I}(t_p^+)\right)\right]
        \end{array}
        $$
        Remarquons que:
        \begin{eqnarray*}
            \mathcal{I}(t_p^+)-\mathcal{I}(t_p^-)&=&\sum_{s\in\gamma^{-1}(\mathcal{B}\cup\mathcal{S}_{\bar{n}})\cap[0, t_p]}\left[\left(\pi-\widehat{\gamma(s)}-2\pi I_{\gamma(s)}\right)-\left(\theta^{\gamma}_{\bar{n}}(s^+) - \theta^{\gamma}_{\bar{n}}(s^-)\right)\right]\\\\
            &&-\sum_{s\in\gamma^{-1}(\mathcal{B}\cup\mathcal{S}_{\bar{n}})\cap[0, t_p[}\left[\left(\pi-\widehat{\gamma(s)}-2\pi I_{\gamma(s)}\right)-\left(\theta^{\gamma}_{\bar{n}}(s^+) - \theta^{\gamma}_{\bar{n}}(s^-)\right)\right]\\\\
            &=&\left(\pi-\widehat{p}-2\pi I_p\right)-\left(\theta^{\gamma}_{\bar{n}}(t_p^+) - \theta^{\gamma}_{\bar{n}}(t_p^-)\right)
        \end{eqnarray*}
        Il vient alors que:
        \begin{eqnarray*}
        id^\partial_{\bar{v}}(p)&=&\displaystyle\frac{1}{2\pi}\left[\pi-\widehat{p}-\left(\theta^{\gamma}_{\bar{n}}(t_p^+) - \theta^{\gamma}_{\bar{n}}(t_p^-)\right)-\left(\pi-\widehat{p}-2\pi I_p\right)+\left(\theta^{\gamma}_{\bar{n}}(t_p^+) - \theta^{\gamma}_{\bar{n}}(t_p^-)\right)\right]\\\\
        &=&\displaystyle\frac{1}{2\pi}\left[\pi-\widehat{p}-\left(\pi-\widehat{p}-2\pi I_p\right)\right]
        \end{eqnarray*}
        Par conséquent, $id^\partial_{\bar{v}}(p)=I_p$.\\
        \item[$\bullet$] Supposons maintenant que $p=\gamma(0)=\gamma(1)$. Puisque $\bar{v}$ est aligné par rapport à $\partial\Omega$, en appliquant le théorème de Poincaré-Hopf à $\bar{v}$, on obtient :
        $$
        \sum_{q\in\Omega} id_{\bar{v}}(q)+\sum_{q\in\partial\Omega\backslash\partial\Omega} id^\partial_{\bar{v}}(q)=\chi(\Omega).
        $$
        % BIEN SUR ICI ON PEUT DIRE QUE LA SOMME DES ID_V= SOMME DES ID_U MAIS CE N'EST PAS DE CA ON A BESOIN. oN A BESOIN DE THETA_U_1-THETA_U_0 ET ON NE SAIT PAS MONTRER QUE SOMME DES ID_U=THETA_U_1-THETA_U_0 A CAUSE DE THETA_U^GAMMA
        Sachant que $\sum_{q\in\Omega\backslash\partial\Omega}id_{\bar{v}}(q)=\theta_{\bar{v}}^\gamma(1)-\theta_{\bar{v}}^\gamma(0)$ (puisqu'il sagit du nombre de fois que $\bar{v}$ tourne sur lui même le long de $\gamma$), il vient que :
        $$
        \phi(\gamma(1))-\phi(\gamma(0))+\left(\theta_{\bar{u}}^\gamma(1)-\theta_{\bar{u}}^\gamma(0)\right)+\sum_{q\in\partial\Omega}id_{\bar{v}}^\gamma(q)=\chi(\Omega)
        $$
        D'après le lemme \ref{lem:marvelous_lemma}, $\phi(\gamma(1))-\phi(\gamma(0))=0$. On a donc:
        $$
        \theta_{\bar{u}}^\gamma(1)-\theta_{\bar{u}}^\gamma(0)+\sum_{q\in\partial\Omega} id^\partial_{\bar{v}}(q)=\chi(\Omega).
        $$
        Or d'après ce qui précède, pour tout $q\in\partial\Omega\backslash\{p\}$, on a $id_{\bar{v}}^\gamma(q)=I_q$. Par conséquent:
        $$
        \theta_{\bar{u}}^\gamma(1)-\theta_{\bar{u}}^\gamma(0)+\sum_{q\in\partial\Omega\backslash\{p\}}I_q + id^\partial_{\bar{v}}(p) + I_p - I_p =\chi(\Omega).
        $$
        Autrement dit,
        $$
        id^\partial_{\bar{v}}(p) - I_p =\chi(\Omega)-\left[\left(\theta_{\bar{u}}^\gamma(1)-\theta_{\bar{u}}^\gamma(0)\right)+\sum_{q\in\partial\Omega} I_q\right]
        $$
        Finalement, en utilisant la condition \eqref{eqn:etude_hypothese_u}, on obtient:
        $$
        id^\partial_v(p)=I_p.
        $$
    \end{itemize}
\end{proof}

Nous donnons maintenant la généralisation du théorème précédent aux domaines non-simplement connexe. On suppose que $\Omega$ est borné et fermé et que $\partial\Omega=\cup_i\Gamma_i$, où $\Gamma_i,~i\in\llbracket 0, n_b-1\rrbracket$ désigne les composantes connexes de $\partial\Omega$ et $n_b$ le nombre de composantes connexes de $\partial\Omega$. Dans ce cas, nous modifions la condition \eqref{eqn:etude_hypothese_u} comme suit:
\begin{equation}
    \left\{
    \begin{array}{lcll}
    \theta_{\bar{u}}^{\gamma_0}(1)-\theta_{\bar{u}}^{\gamma_0}(0)&=&2\pi-2\pi\displaystyle\sum_{p\in(\mathcal{B}\cup\mathcal{S}_{\bar{n}})\cap\Gamma_0}I_p,&\mbox{ sur }\Gamma_0\\\\
    \theta_{\bar{u}}^{\gamma_i}(1)-\theta_{\bar{u}}^{\gamma_i}(0)&=&-2\pi-2\pi\displaystyle\sum_{p\in(\mathcal{B}\cup\mathcal{S}_{\bar{n}})\cap\Gamma_i}I_p,&\mbox{ sur }\Gamma_i,~\forall i\in\llbracket 1, n_b-1\rrbracket.
    \end{array}
    \right.
    \label{eqn:etude_hypothese_u_second}
\end{equation}
où pour tout $i\in\llbracket0, n_b-1\rrbracket$, $\gamma_i$ est une paramétrisation sur $[0, 1]$ de $\Gamma_i$ orientée positivement et vérifiant $\gamma_i(0)=\gamma_i(1)\notin\mathcal{B}\cup\mathcal{S}_{\bar{n}}\cup\mathcal{S}_{\bar{u}}$. L'équation \eqref{eqn:etude_def_phi} devient alors:
\begin{equation}
\left\{
\begin{array}{lcll}
\Delta\phi &=& 0 &\mbox{ dans }\Omega,\\[0.5cm]
\phi(\gamma_i(t))&=&\theta_{\bar{n}}^{\gamma_i}(t)+\mathcal{I}(t)-\theta_{\bar{u}}^{\gamma_i}(t) & \mbox{ sur } \gamma_i^{-1}(\Gamma_i\backslash(\mathcal{B}\cup\mathcal{S}_{\bar{n}}\cup\mathcal{S}_{\bar{u}})),~\forall i\in\llbracket 0, n_b-1\rrbracket.
\end{array}
\right.
\label{eqn:etude_def_phi_second}
\end{equation}
où la fonction $\mathcal{I}$ est donnée pour tout $i\in\llbracket0, n_b-1\rrbracket$, et pour tout $t\in{\gamma_i}^{-1}(\Gamma_i\backslash(\mathcal{B}\cup\mathcal{S}_{\bar{n}}\cup\mathcal{S}_{\bar{u}}))$ par:
$$
\mathcal{I}(t)=\sum_{s\in{\gamma_i}^{-1}((\mathcal{B}\cup\mathcal{S}_{\bar{n}})\cap\Gamma_i)}\left[\left(\pi-\widehat{\gamma_i(s)}-2\pi I_{\gamma_i(s)}\right)-\left(\lim\limits_{r\rightarrow s^+}\theta^{\gamma_i}_{\bar{n}}(r) - \lim\limits_{r\rightarrow s^-}\theta^{\gamma_i}_{\bar{n}}(r)\right)\right]\mathbb{1}_{[0, t]}(s),
$$
avec $\widehat{\gamma_i(s)}$ la mesure de l'ouverture angulaire de la frontière en $\gamma_i(s)$. Nous avons alors le lemme suivant:

\begin{lemma}
    Pour tout $i\in\llbracket0, n_b-1\rrbracket$, $\phi(\gamma_i(1))-\phi(\gamma_i(0))=0$.
    \label{lem:marvelous_lemma_second}
\end{lemma}
\begin{proof}
De façon analogue a ce qui est fait dans la preuve du lemme \ref{lem:marvelous_lemma}, on a pour tout $i\in\llbracket 0, n_b-1\rrbracket$:
$$
\begin{array}{lcl}
    \phi(\gamma_i(1))-\phi(\gamma_i(0))&=&\displaystyle\sum_{i=1}^{n_t}\int_{t_i}^{t_{i+1}}d\theta_{\bar{n}}^{\gamma_i}+\sum_{s\in{\gamma_i}^{-1}((\mathcal{B}\cup\mathcal{S}_{\bar{n}})\cap\Gamma_i)}\left(\pi-\widehat{\gamma_i(s)}\right)\\\\
    &&-\displaystyle\left(\sum_{s\in{\gamma_i}^{-1}((\mathcal{B}\cup\mathcal{S}_{\bar{n}})\cap\Gamma_i)}2\pi I_{\gamma_i(s)}+(\theta_{\bar{u}}^{\gamma_i}(1)-\theta_{\bar{u}}^{\gamma_i}(0))\right)
\end{array}
$$
    où on a posé $t_{n_t+1}:=t_1$. D'après le théorème des tangentes tournantes \cite{hopf1935drehung, rotskoff2010gauss}, on sait que :
    $$
    \left\{
    \begin{array}{ll}
    \displaystyle\sum_{i=1}^{n_t}\left(\theta_{\bar{n}}^{\gamma_0}(t_{i+1})-\theta_{\bar{n}}^{\gamma_0}((t_i)\right)+\sum_{s\in{\gamma_0}^{-1}((\mathcal{B}\cup\mathcal{S}_{\bar{n}})\cap\Gamma_0)}\left(\pi-\widehat{\gamma_0(s)}\right)=2\pi.\\\\
    \displaystyle\sum_{i=1}^{n_t}\left(\theta_{\bar{n}}^{\gamma_i}(t_{j+1})-\theta_{\bar{n}}^{\gamma_i}(t_j)\right)+\sum_{s\in{\gamma_i}^{-1}((\mathcal{B}\cup\mathcal{S}_{\bar{n}})\cap\Gamma_i)}\left(\pi-\widehat{\gamma_i(s)}\right)=-2\pi&\mbox{ si }i\in\llbracket1, n_b-1\rrbracket.
    \end{array}
    \right.
    $$
    On obtient alors en utilisant la condition \eqref{eqn:etude_hypothese_u_second}
    $$
    \left\{
    \begin{array}{lcl}
    \phi(\gamma_0(1))-\phi(\gamma_0(0))&=&2\pi-\displaystyle\sum_{s\in{\gamma_0}^{-1}((\mathcal{B}\cup\mathcal{S}_{\bar{n}})\cap\Gamma_0)}\left(\pi-\widehat{\gamma_0(s)}\right)\\\\
    &&+\displaystyle\sum_{s\in{\gamma_0}^{-1}((\mathcal{B}\cup\mathcal{S}_{\bar{n}})\cap\Gamma_0)}\left(\pi-\widehat{\gamma_0(s)}\right)-2\pi,\\\\
    \phi(\gamma_i(1))-\phi(\gamma_i(0))&=&-2\pi-\displaystyle\sum_{s\in{\gamma_i}^{-1}((\mathcal{B}\cup\mathcal{S}_{\bar{n}})\cap\Gamma_i)}\left(\pi-\widehat{\gamma_i(s)}\right)\\\\
    &&+\displaystyle\sum_{s\in{\gamma_i}^{-1}((\mathcal{B}\cup\mathcal{S}_{\bar{n}})\cap\Gamma_i)}\left(\pi-\widehat{\gamma_i(s)}\right)+2\pi, \mbox{si }i\in\llbracket 1, n_b-1\rrbracket.
    \end{array}
    \right.
    $$
    Autrement dit,
    $$
    \forall i\in\llbracket 0, n_b-1\rrbracket,\mbox{ on a }\phi(\gamma_i(1))-\phi(\gamma_i(0))=0.
    $$
\end{proof}
De manière similaire à la procédure précédente, le champ de croix $\bar{v}$ obtenu à partir de l'opération d'alignement est défini pour tout $p\in\Omega$ par:
\begin{equation}
\bar{v}(p)=
\left\{
\begin{array}{ll}
\mathbf{R}(\phi(p))\bar{u}(p) & \mbox{ si } p\in\Omega\backslash(\mathcal{B}\cup\mathcal{S}_{\bar{n}}\cup\mathcal{S}_{\bar{u}}),\\[0.5cm]
\bar{n}(p) & \mbox{ si } p\in(\mathcal{S}_{\bar{u}}\cap\partial\Omega)\backslash(\mathcal{B}\cup\mathcal{S}_{\bar{n}}),\\[0.5cm]
0 & \mbox{ si } p\in\mathcal{B}\cup\mathcal{S}_{\bar{n}}.
\end{array}
\right.
\label{eqn:etude_def_v_second}
\end{equation}
On a alors le théorème suivant:
\begin{theorem}
\label{thm:theorem3}
Le champ de croix $\bar{v}$ est presque-$\mathcal{C}^1$ sur $\Omega$ et aligné avec $\partial\Omega$. De plus, pour tout $p\in\Omega$, on a:
\begin{equation}
id_{\bar{v}}(p)=
\left\{
\begin{array}{ll}
    id_{\bar{u}}(p) & \mbox{ si } p\in\Omega\backslash\partial\Omega,\\\\
    I_p & \mbox{ sinon }.
\end{array}
\right.
\end{equation}
\end{theorem}
\begin{proof}
    Du fait que la fonction $\phi$ définie par l'équation \eqref{eqn:etude_def_phi_second} est de classe $\mathcal{C}^1$ sur $\Omega\backslash(\mathcal{B}\cup\mathcal{S}_{\bar{n}}\cup(\mathcal{S}_{\bar{u}}\cap\partial\Omega))$, le champ $\bar{v}$ est presque-$\mathcal{C}^1$ sur $\Omega$. Cela découle immédiatement de la proposition \ref{prop:cont1}, étant donné que $\bar{u}$ est presque-$\mathcal{C}^1$ sur $\Omega$.

    De plus, $\bar{v}$ est aligné avec $\partial\Omega$. En effet, $\bar{v}(p)=0$ pour tout $p\in\mathcal{B}\cup\mathcal{S}_{\bar{n}}$ et $\bar{v}(p)=\bar{n}(p)$ pour tout $p\in(\mathcal{S}_{\bar{u}}\cap\partial\Omega)\backslash(\mathcal{B}\cup\mathcal{S}_{\bar{n}})$. Par ailleurs, pour tout $p\in\partial\Omega\backslash(\mathcal{B}\cup\mathcal{S}_{\bar{n}}\cup\mathcal{S}_{\bar{u}})$, $\exists i\in\llbracket0, n_b-1\rrbracket$ tel que $p\in\Gamma_i\backslash(\mathcal{B}\cup\mathcal{S}_{\bar{n}}\cup\mathcal{S}_{\bar{u}})$ et on a:
    $$\theta_{\bar{v}}(\gamma_i(t_p))=\phi(\gamma_i(t_p))+\theta_{\bar{u}}^{\gamma_i}(t_p)+k\frac{\pi}{2}=\theta_{\bar{n}}^{\gamma_i}(t_p)+\mathcal{I}(t_p)-\theta_{\bar{u}}^{\gamma_i}(t_p)+\theta_{\bar{u}}^{\gamma_i}(t_p)+k\frac{\pi}{2},$$
    où $t_p\in[0,1]$ tel que $\gamma_i(t_p)=p$ et $k\in\mathbb{Z}$. Or pour tout $s\in{\gamma_i}^{-1}((\mathcal{B}\cup\mathcal{S}_{\bar{n}})\cap\Gamma_i)\cap[0, t_p]$ on a
    $$
    \lim_{r\rightarrow s^+}\theta^{\gamma_i}_{\bar{n}}(r) - \lim_{r\rightarrow s^-}\theta^{\gamma_i}_{\bar{n}}(r)\equiv(\pi-\widehat{\gamma(s)})\pmod{\frac{\pi}{2}}.
    $$
    Il vient alors que:
    $$
    \theta_{\bar{v}}(\gamma_i(t_p))=\theta_{\bar{n}}^{\gamma_i}(t_p)+\displaystyle\sum_{s\in{\gamma_i}^{-1}((\mathcal{B}\cup\mathcal{S}_{\bar{n}})\cap\Gamma_i)}\left[\left(\pi-\widehat{\gamma_i(s)}-m_s\frac{\pi}{2}\right)-\left(\pi-\widehat{\gamma_i(s)}+k_s\frac{\pi}{2}\right)\right]\mathbb{1}_{[0, t_p]}(s)+k\frac{\pi}{2},
    $$
    où $k_s\in\mathbb{Z}$ et où on a posé $I_{\gamma_i(s)}=m_s/4$ avec $m_s\in\mathbb{Z}$ et $m_s\leq1$. Autrement dit, on a:
    $$
    \theta_{\bar{v}}(\gamma_i(t_p))=\theta_{\bar{n}}^{\gamma_i}(t_p)+\displaystyle\sum_{s\in{\gamma_i}^{-1}(\mathcal{B}\cup\mathcal{S}_{\bar{n}})}\left[-m_s\frac{\pi}{2}+k_s\frac{\pi}{2}\right]\mathbb{1}_{[0, t_p]}(s)+k\frac{\pi}{2}$$
    et donc:
    $$
    \theta_{\bar{v}}(\gamma_i(t_p))\equiv\theta_{\bar{n}}^{\gamma_i}(t_p)\pmod{\frac{\pi}{2}}.
    $$
    Ce qui implique que $\bar{v}(p)=\bar{n}(p)$. On conclut alors que $\bar{v}$ est aligné avec $\partial\Omega$.\\\\
    Nous calculons maintenant pour tout $p\in\Omega$, l'indice de $p$ dans le champ $\bar{v}$. Ainsi, on a:\\
    \begin{itemize}
        \item[$\bullet$] Si $p\in\Omega\backslash\partial\Omega$, $id_{\bar{v}}(p)=id_{\bar{u}}(p)$ selon la proposition \ref{prop:relation_u_Rthetau}, car la fonction $\phi$ définie par l'équation \eqref{eqn:etude_def_phi} est de classe $\mathcal{C}^1$ sur $\Omega\backslash\partial\Omega$.\\
        \item[$\bullet$] Si $p\in\Gamma_i\backslash\{\gamma_i(0)\}$ (sachant que $\gamma_i(0)=\gamma_i(1)$) avec $i\in\llbracket0, n_b-1\rrbracket$, on a:
        $$
        id^\partial_{\bar{v}}(p)=\frac{1}{2\pi}\left[\pi-\widehat{p}+\displaystyle\lim\limits_{s\rightarrow 0}\int_s^{1-s}d\theta_{\bar{v}}^{\mathcal{C}}\right]=\frac{1}{2\pi}\left(\pi-\widehat{p}+(\theta_{\bar{v}}^{\mathcal{C}}(1)-\theta_{\bar{v}}^{\mathcal{C}}(0))\right),
        $$
        où $\mathcal{C}$ est un lacet paramétré sur $[0, 1]$ tel que $\mathcal{C}(0)=p=\mathcal{C}(1)$ et les vecteurs $\mathcal{C}'(0)$ et $\mathcal{C}'(1)$ sont tangents à $\partial\Omega$. De plus, $\mathcal{C}$ n'englobe aucun autre point singulier de $\bar{u}$. Soit $t_p\in]0, 1[$ tel que $\gamma_i(t_p)=p$. On a:
        $$
        id^\partial_{\bar{v}}(p)=\frac{1}{2\pi}\left[\pi-\widehat{p}+\left(\theta_{\bar{v}}^{\gamma_i}(t_p^-)-\theta_{\bar{v}}^{\gamma_i}(t_p^+)\right)\right]
        $$
        où on a noté $\lim_{t\rightarrow t_p^+}\theta^{\gamma_i}_{\bar{v}}(t)=\theta^{\gamma_i}_{\bar{v}}(t_p^+)$ et $\lim_{t\rightarrow t_p^-}\theta^{\gamma_i}_{\bar{v}}(t)=\theta_{\bar{v}}^{\gamma_i}(t_p^-)$. Autrement dit, on a:
        $$
        \begin{array}{lcl}
        id^\partial_{\bar{v}}(p)&=&\displaystyle\frac{1}{2\pi}\left[\pi-\widehat{p}+\left(\phi(\gamma_i(t_p^-))+\theta_{\bar{u}}^{\gamma_i}(t_p^-)-\phi(\gamma_i(t_p^+))-\theta_{\bar{u}}^{\gamma_i}(t_p^+)\right)\right]\\\\
        &=&\displaystyle\frac{1}{2\pi}\left[\pi-\widehat{p}+\left(\theta_{\bar{n}}^{\gamma_i}(t_p^-)-\theta_{\bar{n}}^{\gamma_i}(t_p^+)\right)+\left(\mathcal{I}(t_p^-)-\mathcal{I}(t_p^+)\right)\right]
        \end{array}
        $$
        Remarquons que:
        $$
        \begin{array}{ll}
            \mathcal{I}(t_p^+)-\mathcal{I}(t_p^-)=&\displaystyle\sum_{s\in{\gamma_i}^{-1}((\mathcal{B}\cup\mathcal{S}_{\bar{n}})\cap\Gamma_i)\cap[0, t_p]}\left[\left(\pi-\widehat{\gamma_i(s)}-2\pi I_{\gamma_i(s)}\right)-\left(\theta^{\gamma_i}_{\bar{n}}(s^+) - \theta^{\gamma_i}_{\bar{n}}(s^-)\right)\right]\\\\
            &-\displaystyle\sum_{s\in{\gamma_i}^{-1}((\mathcal{B}\cup\mathcal{S}_{\bar{n}})\cap\Gamma_i)\cap[0, t_p[}\left[\left(\pi-\widehat{\gamma_i(s)}-2\pi I_{\gamma_i(s)}\right)-\left(\theta^{\gamma_i}_{\bar{n}}(s^+) - \theta^{\gamma_i}_{\bar{n}}(s^-)\right)\right]\\\\
            \mathcal{I}(t_p^+)-\mathcal{I}(t_p^-)=&\left(\pi-\widehat{p}-2\pi I_p\right)-\left(\theta^{\gamma_i}_{\bar{n}}(t_p^+) - \theta^{\gamma_i}_{\bar{n}}(t_p^-)\right)
        \end{array}
        $$
        Il vient alors que:
        \begin{eqnarray*}
        id^\partial_{\bar{v}}(p)&=&\displaystyle\frac{1}{2\pi}\left[\pi-\widehat{p}-\left(\theta^{\gamma_i}_{\bar{n}}(t_p^+) - \theta^{\gamma_i}_{\bar{n}}(t_p^-)\right)-\left(\pi-\widehat{p}-2\pi I_p\right)+\left(\theta^{\gamma_i}_{\bar{n}}(t_p^+) - \theta^{\gamma_i}_{\bar{n}}(t_p^-)\right)\right]\\\\
        &=&\displaystyle\frac{1}{2\pi}\left[\pi-\widehat{p}-\left(\pi-\widehat{p}-2\pi I_p\right)\right]
        \end{eqnarray*}
        Par conséquent, $id^\partial_{\bar{v}}(p)=I_p$.\\
        \item[$\bullet$] Supposons maintenant que $p\in\Gamma_i$ tel que $p=\gamma_i(0)=\gamma_i(1)$ avec $i\in\llbracket0, n_b-1\rrbracket$. De ce qui précède, il ressort que $\bar{v}$ est aligné par rapport à $\partial\Omega$ donc aligné sur $\Gamma_i$. Par conséquent, d'après le théorème des tangentes tournantes \cite{hopf1935drehung, rotskoff2010gauss}, on a:
        $$
        \left\{
        \begin{array}{ll}
         \displaystyle\int_0^1 d\theta_{\bar{v}}^{\gamma_0}+\sum_{s\in{\gamma_0}^{-1}((\mathcal{B}\cup\mathcal{S}_{\bar{n}})\cap\Gamma_0)}\left(\pi-\widehat{\gamma_0(s)}\right)=2\pi&\mbox{ si }i=0,\\\\
        \displaystyle\int_0^1 d\theta_{\bar{v}}^{\gamma_i}+\sum_{s\in{\gamma_i}^{-1}((\mathcal{B}\cup\mathcal{S}_{\bar{n}})\cap\Gamma_i)}\left(\pi-\widehat{\gamma_i(s)}\right)=-2\pi&\mbox{ si }i\in\llbracket 1, n_b-1\rrbracket.
        \end{array}
        \right.
        $$
        Ce qui est équivalent à
        $$
        \left\{
        \begin{array}{lcl}
         2\pi&=&\displaystyle\int_0^1 d\theta_{\bar{v}}^{\gamma_0}+\displaystyle\sum_{s\in{\gamma_0}^{-1}((\mathcal{B}\cup\mathcal{S}_{\bar{n}})\cap\Gamma_0)}\left(\theta^{\gamma_0}_{\bar{v}}(s^+)-\theta^{\gamma_0}_{\bar{v}}(s^-)\right)\\\\
         &&+\displaystyle\sum_{s\in{\gamma_0}^{-1}((\mathcal{B}\cup\mathcal{S}_{\bar{n}})\cap\Gamma_0)}\left(\pi-\widehat{\gamma_0(s)}\right)-\displaystyle\sum_{s\in{\gamma_0}^{-1}((\mathcal{B}\cup\mathcal{S}_{\bar{n}})\cap\Gamma_0)}\left(\theta^{\gamma_0}_{\bar{v}}(s^+)-\theta^{\gamma_0}_{\bar{v}}(s^-)\right),\\\\
         &&\mbox{ si }i=0,\\\\
        -2\pi&=&\displaystyle\int_0^1 d\theta_{\bar{v}}^{\gamma_i}+\displaystyle\sum_{s\in{\gamma_i}^{-1}((\mathcal{B}\cup\mathcal{S}_{\bar{n}})\cap\Gamma_i)}\left(\theta^{\gamma_i}_{\bar{v}}(s^+) - \theta^{\gamma_i}_{\bar{v}}(s^-)\right)\\\\
        &&+\displaystyle\sum_{s\in{\gamma_i}^{-1}((\mathcal{B}\cup\mathcal{S}_{\bar{n}})\cap\Gamma_i)}\left(\pi-\widehat{\gamma_i(s)}\right)-\displaystyle\sum_{s\in{\gamma_i}^{-1}((\mathcal{B}\cup\mathcal{S}_{\bar{n}})\cap\Gamma_i)}\left(\theta^{\gamma_i}_{\bar{v}}(s^+)-\theta^{\gamma_i}_{\bar{v}}(s^-)\right),\\\\
        &&\mbox{ si }i\in\llbracket 1, n_b-1\rrbracket.
        \end{array}
        \right.
        $$
        Autrement dit,
        $$
        \left\{
        \begin{array}{ll}
        \theta_{\bar{v}}^{\gamma_0}(1)-\theta_{\bar{v}}^{\gamma_0}(0)+
         \displaystyle\sum_{s\in{\gamma_0}^{-1}((\mathcal{B}\cup\mathcal{S}_{\bar{n}})\cap\Gamma_0)}id^\partial_{\bar{v}}(\gamma_0(s))=2\pi&\mbox{ si }i=0,\\\\
        \theta_{\bar{v}}^{\gamma_i}(1)-\theta_{\bar{v}}^{\gamma_i}(0)+
         \displaystyle\sum_{s\in{\gamma_i}^{-1}((\mathcal{B}\cup\mathcal{S}_{\bar{n}})\cap\Gamma_i)}id^\partial_{\bar{v}}(\gamma_i(s))=-2\pi&\mbox{ si }i\in\llbracket 1, n_b-1\rrbracket.
        \end{array}
        \right.
        $$
        On a donc:
        $$
        \left\{
        \begin{array}{r}
        \phi(\gamma_0(1))-\phi(\gamma_0(0))+\left(\theta_{\bar{u}}^{\gamma_0}(1)-\theta_{\bar{u}}^{\gamma_0}(0)\right)+
         \displaystyle\sum_{s\in{\gamma_0}^{-1}((\mathcal{B}\cup\mathcal{S}_{\bar{n}})\cap\Gamma_0)}id^\partial_{\bar{v}}(\gamma_0(s))=2\pi\\\\
         \mbox{ si }i=0,\\\\
        \phi(\gamma_i(1))-\phi(\gamma_i(0))+\left(\theta_{\bar{u}}^{\gamma_i}(1)-\theta_{\bar{u}}^{\gamma_i}(0)\right)+
         \displaystyle\sum_{s\in{\gamma_i}^{-1}((\mathcal{B}\cup\mathcal{S}_{\bar{n}})\cap\Gamma_i)}id^\partial_{\bar{v}}(\gamma_i(s))=-2\pi\\\\
         \mbox{ si }i\in\llbracket 1, n_b-1\rrbracket.
        \end{array}
        \right.
        $$
        D'après le lemme \ref{lem:marvelous_lemma_second}, $\phi(\gamma_i(1))-\phi(\gamma_i(0))=0$ pour tout $i\in\llbracket0, n_b-1\rrbracket$. On a donc:
        $$
        \left\{
        \begin{array}{ll}
        \theta_{\bar{u}}^{\gamma_0}(1)-\theta_{\bar{u}}^{\gamma_0}(0)+
         \displaystyle\sum_{q\in(\mathcal{B}\cup\mathcal{S}_{\bar{n}})\cap\Gamma_0}id^\partial_{\bar{u}}(q)=2\pi&\mbox{ si }i=0,\\\\
        \theta_{\bar{u}}^{\gamma_i}(1)-\theta_{\bar{u}}^{\gamma_i}(0)+
         \displaystyle\sum_{q\in(\mathcal{B}\cup\mathcal{S}_{\bar{n}})\cap\Gamma_i}id^\partial_{\bar{v}}(q)=-2\pi&\mbox{ si }i\in\llbracket 1, n_b-1\rrbracket.
        \end{array}
        \right.
        $$
        Or d'après ce qui précède, pour tout $q\in\partial\Omega\backslash\{p\}$, on a $id_{\bar{v}}^\gamma(q)=I_q$. Par conséquent:
        $$
        \left\{
        \begin{array}{ll}
        \theta_{\bar{u}}^{\gamma_0}(1)-\theta_{\bar{u}}^{\gamma_0}(0)+
         \displaystyle\sum_{q\in\Gamma_0\backslash\{p\}}
         I_q+ id^\partial_{\bar{v}}(p)+I_p-I_p=2\pi&\mbox{ si }i=0,\\\\
        \theta_{\bar{u}}^{\gamma_i}(1)-\theta_{\bar{u}}^{\gamma_i}(0)+
         \displaystyle\sum_{q\in\Gamma_i\backslash\{p\}}I_q+ id^\partial_{\bar{v}}(p)+I_p-I_p=-2\pi&\mbox{ si }i\in\llbracket 1, n_b-1\rrbracket.
        \end{array}
        \right.
        $$
        Autrement dit,
         $$
        \left\{
        \begin{array}{ll}
         id^\partial_{\bar{v}}(p)-I_p=2\pi-\left[\left(\theta_{\bar{u}}^{\gamma_0}(1)-\theta_{\bar{u}}^{\gamma_0}(0)\right)+\displaystyle\sum_{q\in\Gamma_0} I_q\right]&\mbox{ si }i=0,\\\\
         id^\partial_{\bar{v}}(p)-I_p=-2\pi-\left[\left(\theta_{\bar{u}}^{\gamma_i}(1)-\theta_{\bar{u}}^{\gamma_i}(0)\right)+\displaystyle\sum_{q\in\Gamma_i} I_q\right]&\mbox{ si }i\in\llbracket 1, n_b-1\rrbracket.
        \end{array}
        \right.
        $$
        Finalement, en utilisant la condition \eqref{eqn:etude_hypothese_u_second}, on obtient:
        $$
        id^\partial_v(p)=I_p.
        $$
    \end{itemize}
\end{proof}



\section{Analyse de convergence}

\section{Génération de champs de croix}


\newpage




Dans la section précédente, nous avons exposé des algorithmes qui, sous certaines conditions, permettent de subdiviser un domaine en régions à quatres côtés. Dans cette partie, nous allons présenter plusieurs résultats visant à garantir l'efficacité de ces méthodes. Commençons par le cas où le champ de croix est aligné avec le bord du domaine. Nous avons le résultat suivant:
\begin{theorem}
\label{thm:theorem1}
Soit $\Omega$ un domaine borné et fermé dans $\mathbb{R}^2$ avec une frontière régulière par morceau et soit $\bar{u}$ un champ de croix presque-$\mathcal{C}^1$ aligné avec $\partial\Omega$ tel que $0<Card(\mathcal{S}_{\bar{u}})<\infty$ et pour tout $p\in\Omega$, $id_{\bar{u}}(p)=k/4$ où $k\in\mathbb{Z}$ et $k\leq 1$. Si l'algorithme de partitionnement \ref{alg:algo_main} appliqué à $\bar{u}$ converge alors le partitionnement résultant est une décomposition de $\Omega$ en régions à quatre côtés.
\end{theorem}

\begin{proof}
Selon les propositions \ref{prop:stream_from_interior_sing} et \ref{prop:stream_from_bord_sing}, chaque point singulier de $u$ donne lieu à un nombre fini de séparatrices. Étant donné que ces séparatrices ne convergent pas vers des cycles limites, elles doivent soit se terminer en un point singulier, soit intersecter la frontière de $\Omega$. En conséquence, les séparatrices de $\bar{u}$ divisent $\Omega$ en régions qui ne contiennent aucun point singulier et qui sont délimitées par des séparatrices.

Soit $\mathcal{R}$ l'une de ces régions. Selon le Théorème de Poincaré-Hopf, on a $\chi(\mathcal{R})=\sum_{i=1}^{n_c} id_{\bar{u}}(c_i)$, où $(c_i)_{i\in\llbracket1,n_c\rrbracket}$ désigne les coins de $\mathcal{R}$ (c'est à dire les points d'intersections des séparatrices formant la région $\mathcal{R}$) et $n_c$ est le nombre de coins. $\chi(\mathcal{R})\geq0$ car, d'après la proposition \ref{prop:loveprop}, nous avons $id_{\bar{u}}(c_i)=1/4>0$ pour chaque coin $c_i$ de $\mathcal{R}$. De plus, on sait que $\chi(\mathcal{R}) = 2 - 2g - b$, où $g$ représente le genre de $\mathcal{R}$ et $b$ le nombre de composantes connexes de $\partial\mathcal{R}$. Comme $\mathcal{R}$ est une partie du plan, on a $g = 0$, ce qui implique que $\chi(\mathcal{R})\in\{0, 1\}$.
\begin{itemize}
\item $\chi(\mathcal{R})=0$: dans ce cas on a $n_c=0$ et $\mathcal{R}=\Omega$. Cela mène à une contradiction puisque, selon les hypothèses du théorème on a $Card(\mathcal{S}_{\bar{u}})>0$.
\item $\chi(\mathcal{R})=1$: cela implique que :
$$1=\sum_{i=1}^{n_c}id_u(c_i)=\sum_{i=1}^{n_c}\frac{1}{4}\Longrightarrow n_c=4.$$
Ainsi, $\mathcal{R}$ est une région à quatre côtés.
\end{itemize}
\end{proof}

definition champ de croix avec fibré tangent\\
Les notions de points singuliers et d'indices restent inchangé.\\
definition de ces notions sur $\Omega_h$\\
Dans toute la suite, nous nous restreignons aux variété ave bord ou s\\


champ aligné ou varité sans bord\\



Nous proposons dans cette partie une extension de la méthode présenté dans les parties précédentes aux surfaces courbes. Pour se faire, nous reviendrons dans ce contexte sur les grandes étapes de la méthodes.
\[\]
La Difficulté pricipale qui va se poser lorsqu'on se place dans le cadre des surface courbe est le calcul des agles des croix.
\[\]
Commencçons par passer en revue dans le cadre des surfaces courbes les différents résultats obtenus dans le chapitre \ref{chap:theoritical}
\[\]

On parlera de $\bar{u}$ et $\bar{u}_h$ en même temps..

Champ de croix\\
Index avec angle defect\\
Streamline sur maillage\\
Theoreme 1, 2, 3\\


Dans la section précédente, nous avons exposé des algorithmes qui, sous certaines conditions, permettent de subdiviser un domaine en régions à quatres côtés. Dans cette partie, nous allons présenter plusieurs résultats visant à garantir l'efficacité de ces méthodes. Commençons par le cas où le champ de croix est aligné avec le bord du domaine. Nous avons le résultat suivant:
\begin{theorem}

Soit $\Omega$ un domaine borné et fermé dans $\mathbb{R}^2$ avec une frontière régulière par morceau et soit $\bar{u}$ un champ de croix presque-$\mathcal{C}^1$ aligné avec $\partial\Omega$ tel que $0<Card(\mathcal{S}_{\bar{u}})<\infty$ et pour tout $p\in\Omega$, $id_{\bar{u}}(p)=k/4$ où $k\in\mathbb{Z}$ et $k\leq 1$. Si l'algorithme de partitionnement \ref{alg:algo_main} appliqué à $\bar{u}$ converge alors le partitionnement résultant est une décomposition de $\Omega$ en régions à quatre côtés.
\end{theorem}




Soit $\Omega$ un domaine fermé et borné dans $\mathbb{R}^2$ avec une frontière lisse par morceaux. Supposons dans un premier temps que $\Omega$ est simplement connexe. Soit $\mathcal{B}\subset\partial\Omega$ un ensemble de point isolé de $\partial\Omega$ et $I_p$ un paramètre associé à chaque point $p\in\Omega$ tel que:
\begin{equation}
I_p=
\left\{
\begin{array}{ll}
\displaystyle\frac{k}{4}\mbox{ avec }k\in\mathbb{Z}\mbox{ et }k\leq 1& \mbox{ si } p\in\mathcal{B}\\[0.5cm]
0& \mbox{ sinon }
\end{array}
\right.
%\label{eqn:etude_def_I}
\end{equation}
Soit $\bar{u}$ un champ de croix presque-$\mathcal{C}^1$ défini sur $\Omega$, non nécessairement aligné sur $\partial\Omega$, tel que $0<Card(\mathcal{S}_{\bar{u}})<\infty$ et pour tout point $p\in\mathcal{S}_{\bar{u}}\backslash\partial\Omega$, $id_{\bar{u}}(p)=k/4$, avec $k\in\mathbb{Z}$ et $k\leq 1$. On suppose de plus qu'il existe $\theta_{\bar{u}}^\gamma$ un relèvement continu de $\bar{u}$ le long de $\gamma$ tel que:
\begin{equation}
    %\label{eqn:etude_hypothese_u}
    \theta_{\bar{u}}^\gamma(1)-\theta_{\bar{u}}^\gamma(0)=2\pi\chi(\Omega)-2\pi\sum_{p\in\mathcal{B}}I_p.
\end{equation}
où $\gamma$ est une paramétrisation sur $[0, 1]$ de $\partial\Omega$ orientée positivement et vérifiant $\gamma(0)=\gamma(1)\notin\mathcal{B}\cup\mathcal{S}_{\bar{N}}\cup\mathcal{S}_{\bar{u}}$.
Soit $\phi$, la fonction définie par l'équation de Laplace suivant:
\begin{equation}
\left\{
\begin{array}{lcll}
\Delta\phi &=& 0 &\mbox{ dans }\Omega,\\[0.5cm]
\phi(\gamma(t))&=&\theta_{\bar{N}}^\gamma(t)+\mathcal{I}(t)-\theta_{\bar{u}}(\gamma(t))& \mbox{ sur } \gamma^{-1}(\partial\Omega\backslash(\mathcal{B}\cup\mathcal{S}_{\bar{N}}\cup\mathcal{S}_{\bar{u}})),
\end{array}
\right.
%\label{eqn:etude_def_phi}
\end{equation}
où la fonction $\mathcal{I}$ est donnée pour tout $t\in\gamma^{-1}(\partial\Omega\backslash(\mathcal{B}\cup\mathcal{S}_{\bar{N}}\cup\mathcal{S}_{\bar{u}}))$ par:
$$
\mathcal{I}(t)=\sum_{s\in\gamma^{-1}(\mathcal{B}\cup\mathcal{S}_{\bar{N}})}\left[\left(\pi-\widehat{\gamma(s)}-2\pi I_{\gamma(s)}\right)-\left(\lim\limits_{r\rightarrow s^+}\theta^{\gamma}_{\bar{N}}(r) - \lim\limits_{r\rightarrow s^-}\theta^{\gamma}_{\bar{N}}(r)\right)\right]\mathbb{1}_{[0, t]}(s),
$$
avec $\widehat{\gamma(s)}$ la mesure de l'ouverture angulaire de la frontière en $\gamma(s)$. Nous avons alors le lemme suivant:

\begin{lemma}
    %\label{lem:marvelous_lemma}
    $\phi(\gamma(1))-\phi(\gamma(0))=0$.
\end{lemma}


L'opération d'alignement consiste à calculer le champ de croix $\bar{v}$ défini pour tout $p\in\Omega$ par:
\begin{equation}
\bar{v}(p)=
\left\{
\begin{array}{ll}
\mathbf{R}(\phi(p))\bar{u}(p) & \mbox{ si } p\in\Omega\backslash(\mathcal{B}\cup\mathcal{S}_{\bar{N}}\cup\mathcal{S}_{\bar{u}}),\\[0.5cm]
\bar{N}(p) & \mbox{ si } p\in(\mathcal{S}_{\bar{u}}\cap\partial\Omega)\backslash(\mathcal{B}\cup\mathcal{S}_{\bar{N}}),\\[0.5cm]
0 & \mbox{ si } p\in\mathcal{B}\cup\mathcal{S}_{\bar{N}}.
\end{array}
\right.
%\label{eqn:etude_def_v}
\end{equation}
Nous avons alors le théorème suivant:
\begin{theorem}
%\label{thm:theorem2}
Le champ de croix $\bar{v}$ est presque-$\mathcal{C}^1$ sur $\Omega$ et aligné avec $\partial\Omega$. De plus, pour tout $p\in\Omega$, on a:
\begin{equation}
id_{\bar{v}}(p)=
\left\{
\begin{array}{ll}
    id_{\bar{u}}(p) & \mbox{ si } p\in\Omega\backslash\partial\Omega,\\\\
    I_p & \mbox{ sinon }.
\end{array}
\right.
\end{equation}
\end{theorem}



Nous donnons maintenant la généralisation du théorème précédent aux domaines non-simplement connexe. On suppose que $\Omega$ est borné et fermé et que $\partial\Omega=\cup_i\Gamma_i$, où $\Gamma_i,~i\in\llbracket 0, n_b-1\rrbracket$ désigne les composantes connexes de $\partial\Omega$ et $n_b$ le nombre de composantes connexes de $\partial\Omega$. Dans ce cas, nous modifions la condition \eqref{eqn:etude_hypothese_u} comme suit:
\begin{equation}
    \left\{
    \begin{array}{lcll}
    \theta_{\bar{u}}^{\gamma_0}(1)-\theta_{\bar{u}}^{\gamma_0}(0)&=&2\pi-2\pi\displaystyle\sum_{p\in(\mathcal{B}\cup\mathcal{S}_{\bar{N}})\cap\Gamma_0}I_p,&\mbox{ sur }\Gamma_0\\\\
    \theta_{\bar{u}}^{\gamma_i}(1)-\theta_{\bar{u}}^{\gamma_i}(0)&=&-2\pi-2\pi\displaystyle\sum_{p\in(\mathcal{B}\cup\mathcal{S}_{\bar{N}})\cap\Gamma_i}I_p,&\mbox{ sur }\Gamma_i,~\forall i\in\llbracket 1, n_b-1\rrbracket.
    \end{array}
    \right.
    \label{eqn:etude_hypothese_u_second}
\end{equation}
où pour tout $i\in\llbracket0, n_b-1\rrbracket$, $\gamma_i$ est une paramétrisation sur $[0, 1]$ de $\Gamma_i$ orientée positivement et vérifiant $\gamma_i(0)=\gamma_i(1)\notin\mathcal{B}\cup\mathcal{S}_{\bar{N}}\cup\mathcal{S}_{\bar{u}}$. L'équation \eqref{eqn:etude_def_phi} devient alors:
\begin{equation}
\left\{
\begin{array}{lcll}
\Delta\phi &=& 0 &\mbox{ dans }\Omega,\\[0.5cm]
\phi(\gamma_i(t))&=&\theta_{\bar{N}}^{\gamma_i}(t)+\mathcal{I}(t)-\theta_{\bar{u}}^{\gamma_i}(t) & \mbox{ sur } \gamma_i^{-1}(\Gamma_i\backslash(\mathcal{B}\cup\mathcal{S}_{\bar{N}}\cup\mathcal{S}_{\bar{u}})),~\forall i\in\llbracket 0, n_b-1\rrbracket.
\end{array}
\right.
%\label{eqn:etude_def_phi_second}
\end{equation}
où la fonction $\mathcal{I}$ est donnée pour tout $i\in\llbracket0, n_b-1\rrbracket$, et pour tout $t\in{\gamma_i}^{-1}(\Gamma_i\backslash(\mathcal{B}\cup\mathcal{S}_{\bar{N}}\cup\mathcal{S}_{\bar{u}}))$ par:
$$
\mathcal{I}(t)=\sum_{s\in{\gamma_i}^{-1}((\mathcal{B}\cup\mathcal{S}_{\bar{N}})\cap\Gamma_i)}\left[\left(\pi-\widehat{\gamma_i(s)}-2\pi I_{\gamma_i(s)}\right)-\left(\lim\limits_{r\rightarrow s^+}\theta^{\gamma_i}_{\bar{N}}(r) - \lim\limits_{r\rightarrow s^-}\theta^{\gamma_i}_{\bar{N}}(r)\right)\right]\mathbb{1}_{[0, t]}(s),
$$
avec $\widehat{\gamma_i(s)}$ la mesure de l'ouverture angulaire de la frontière en $\gamma_i(s)$. Nous avons alors le lemme suivant:

\begin{lemma}
    Pour tout $i\in\llbracket0, n_b-1\rrbracket$, $\phi(\gamma_i(1))-\phi(\gamma_i(0))=0$.
    %\label{lem:marvelous_lemma_second}
\end{lemma}


De manière similaire à la procédure précédente, le champ de croix $\bar{v}$ obtenu à partir de l'opération d'alignement est défini pour tout $p\in\Omega$ par:
\begin{equation}
\bar{v}(p)=
\left\{
\begin{array}{ll}
\mathbf{R}(\phi(p))\bar{u}(p) & \mbox{ si } p\in\Omega\backslash(\mathcal{B}\cup\mathcal{S}_{\bar{N}}\cup\mathcal{S}_{\bar{u}}),\\[0.5cm]
\bar{N}(p) & \mbox{ si } p\in(\mathcal{S}_{\bar{u}}\cap\partial\Omega)\backslash(\mathcal{B}\cup\mathcal{S}_{\bar{N}}),\\[0.5cm]
0 & \mbox{ si } p\in\mathcal{B}\cup\mathcal{S}_{\bar{N}}.
\end{array}
\right.
%\label{eqn:etude_def_v_second}
\end{equation}
On a alors le théorème suivant:
\begin{theorem}
%\label{thm:theorem3}
Le champ de croix $\bar{v}$ est presque-$\mathcal{C}^1$ sur $\Omega$ et aligné avec $\partial\Omega$. De plus, pour tout $p\in\Omega$, on a:
\begin{equation}
id_{\bar{v}}(p)=
\left\{
\begin{array}{ll}
    id_{\bar{u}}(p) & \mbox{ si } p\in\Omega\backslash\partial\Omega,\\\\
    I_p & \mbox{ sinon }.
\end{array}
\right.
\end{equation}
\end{theorem}


En cas de non-conformité du champ de croix $\bar{u}$ à la condition \eqref{eqn:etude_hypothese_u_second}, nous introduisons un processus de correction visant à obtenir un champ de croix satisfaisant cette condition. Considérons $\bar{u}$ comme un champ de croix presque-$\mathcal{C}^1$ défini sur $\Omega$, sans nécessité d'alignement sur $\partial\Omega$, tel que $0<Card(\mathcal{S}_{\bar{u}})<\infty$ et pour tout point $p\in\mathcal{S}_{\bar{u}}\backslash\partial\Omega$, $id_{\bar{u}}(p)=k/4$, avec $k\in\mathbb{Z}$ et $k\leq 1$. On suppose de plus qu'il existe $\theta_{\bar{u}}^\gamma$ un relèvement continu de $\bar{u}$ tel que:
\begin{equation}
    \displaystyle\sum_{i=0}^{n_b-1}\left(\theta_{\bar{u}}^{\gamma_i}(1)-\theta_{\bar{u}}^{\gamma_i}(0)\right)=2\pi\chi(\Omega)-2\pi\sum_{p\in\mathcal{B}\cup\mathcal{S}_{\bar{N}}}I_p.
\end{equation}
où pour tout $i\in\llbracket0, n_b-1\rrbracket$, $\gamma_i$ est une paramétrisation sur $[0, 1]$ de $\Gamma_i$ orientée positivement et vérifiant $\gamma_i(0)=\gamma_i(1)\notin\mathcal{B}\cup\mathcal{S}_{\bar{N}}\cup\mathcal{S}_{\bar{u}}$.
Soit $\bar{w}$ un champ de croix presque-$\mathcal{C}^1$ définit sur $\Omega$ et vérifiant:
\begin{equation}
    \left\{
    \begin{array}{lcll}
    \theta_{\bar{w}}^{\gamma_0}(1)-\theta_{\bar{w}}^{\gamma_0}(0)&=&\theta_{\bar{u}}^{\gamma_0}(0)-\theta_{\bar{u}}^{\gamma_0}(1)+2\pi\left(1-\displaystyle\sum_{p\in(\mathcal{B}\cup\mathcal{S}_{\bar{N}})\cap\Gamma_0}I_p\right),&\mbox{ sur }\Gamma_0\\\\
    \theta_{\bar{w}}^{\gamma_i}(1)-\theta_{\bar{w}}^{\gamma_i}(0)&=&\theta_{\bar{u}}^{\gamma_i}(0)-\theta_{\bar{u}}^{\gamma_i}(1)-2\pi\left(1+\displaystyle\sum_{p\in(\mathcal{B}\cup\mathcal{S}_{\bar{N}})\cap\Gamma_i}I_p\right),&\mbox{ sur }\Gamma_i,\\\\
    &&&~\forall i\in\llbracket 1, n_b-1\rrbracket.
    \end{array}
    \right.
    %\label{eqn:etude_hypothese_w}
\end{equation}

Dans la section \ref{sec:discussion}, nous aborderons les diverses approches pour la construction d'un tel champ de croix

\begin{proposition}
    Le champ de croix $\bar{u}_c$ défini par $\bar{u}_c:=\mathbf{R}(\theta_{\bar{w}})\bar{u}$ est presque-$\mathcal{C}^1$ sur $\Omega$ et vérifie la condition \eqref{eqn:etude_hypothese_u_second}.
\end{proposition}


Après cela, l'opération d'alignement peut être exécutée sur le champ $\bar{u}_c$, aboutissant au champ de croix $\bar{v}$ défini pour tout $p\in\Omega$ par :
\begin{equation}
\bar{v}(p)=
\left\{
\begin{array}{ll}
\mathbf{R}(\phi(p))\bar{u}_c(p) & \mbox{ si } p\in\Omega\backslash(\mathcal{B}\cup\mathcal{S}_{\bar{N}}\cup(\mathcal{S}_{\bar{u}_c}\cap\partial\Omega)),\\[0.5cm]
\bar{N}(p) & \mbox{ si } p\in(\mathcal{S}_{\bar{u}_c}\cap\partial\Omega)\backslash(\mathcal{B}\cap\mathcal{S}_{\bar{N}}),\\[0.5cm]
0 & \mbox{ si } p\in\mathcal{B}\cup\mathcal{S}_{\bar{N}}.
\end{array}
\right.
%\label{eqn:etude_def_v_third}
\end{equation}

où $\phi$ est donné par:
\begin{equation}
\left\{
\begin{array}{lcll}
\Delta\phi &=& 0 &\mbox{ dans }\Omega,\\[0.5cm]
\phi(\gamma_i(t))&=&\theta_{\bar{N}}^{\gamma_i}(t)+\mathcal{I}(t)-\theta_{\bar{u}_c}^{\gamma_i}(t) & \mbox{ sur } \gamma_i^{-1}(\Gamma_i\backslash(\mathcal{B}\cup\mathcal{S}_{\bar{N}}\cup\mathcal{S}_{\bar{u}_c})),~\forall i\in\llbracket 0, n_b-1\rrbracket.
\end{array}
\right.
%\label{eqn:etude_def_phi_third}
\end{equation}

On a alors le théorème suivant:
\begin{theorem}
%\label{thm:theorem4}
Le champ de croix $\bar{v}$ est presque-$\mathcal{C}^1$ sur $\Omega$ et aligné avec $\partial\Omega$. Si de plus $\mathcal{S}_{\bar{w}}=\emptyset$, alors on a $\mathcal{S}_{\bar{v}}\backslash\partial\Omega=\mathcal{S}_{\bar{u}}\backslash\partial\Omega$ et pour tout $p\in\Omega$, on a:
\begin{equation}
id_{\bar{v}}(p)=
\left\{
\begin{array}{ll}
    id_{\bar{u}}(p) & \mbox{ si } p\in\Omega\backslash\partial\Omega,\\\\
    I_p & \mbox{ sinon }.
\end{array}
\right.
\end{equation}
\end{theorem}



Champ de croix\\
Index\\
Separatrices\\
Operation d'alignement\\
Les théorèmes
generation des champs de croix


Dans \cite{crane2010trivial, de2010trivial} les auteurs proposent un algorithme pour calculer des connexions triviales avec des singularités prescrites sur des surfaces discrètes générales. Nous adaptons cet algorithme pour construire des champs de croix défini sur les sommets d'un maillage triangulaire.

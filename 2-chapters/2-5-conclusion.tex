\chapter{Conclusion et perspectives}
\label{chap:conclusion}
%\minitoc

\section*{Conclusison}

Dans cette thèse, nous avons voulu\\
investiguer et mieux comprendre la génération de maillage quadrilatéral à partir de champ de croix.\\
Pour se faire, nous avons défini un cadre formelle



\section*{Perspectives}

Une affaire en 4 dimension

Mesh $Q_2$\\
\[\]
Homogénéisation e\\
l'idéal serait d'avoir une equation dont la resolution trouverait automatique une répartition des points singulier dont les raport de force soit equilibré permettant aisint d'obtenir un partiionnelent le plus homogène possible
\[\]

priodicité

cycle limite\\
\[\]

adaptation\\
       Les propriétés des lignes de champ guident la densité et la direction du maillage, reflétant les caractéristiques du champ vectoriel.\\
       illustré le champ de croix dependant de la methrique de keenan crane
\[\]

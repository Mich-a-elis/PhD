 
\begin{vcenterpage}

\noindent\rule[2pt]{\textwidth}{0.5pt}

{\textbf{Résumé ---}}
L'objectif de cette thèse est d'apporter de nouvelles solutions pour améliorer les performances lors de la résolution numérique d'équations aux dérivées partielles en étudiant une méthode de génération de maillages quadrilatéraux basée sur des champs de croix.

Notre approche repose sur l'alignement d'un champ de croix donné par rapport au bord du domaine de calcul, suivi du partitionnement de ce dernier en régions à quatre côtés que l'on maillera ensuite en quadrilatères. Cela nous permet de générer un maillage structuré par bloc tout en préservant la position des singularités du champ de croix initial. Pour ce faire, nous mettons en place une étude théorique des champs de croix nous permettant d'identifier les conditions sous lesquelles un champ de croix donné permet effectivement de partitionner et de mailler le domaine sur lequel il est défini.

Cette méthode est d'abord présentée pour les domaines simplement connexes, puis généralisée aux domaines non-simplement connexes. Elle nous permet de gérer les domaines constitués de plusieurs matériaux ainsi que les points singuliers de bord qui, en pratique, permettent de délimiter des portions de la frontière du domaine pour prendre en compte des conditions aux limites mixtes dans le cadre des simulations numériques. Enfin, nous mettons en place la discrétisation de la méthode sur des maillages triangulaires, puis nous proposons une généralisation à des surfaces courbes dans l'espace.\\

{\textbf{Mots clés:}}
    Champ de croix, maillage quadrilatéral, maillage en blocs structurés, energie de Ginzburg Landau.
\\
\noindent\rule[2pt]{\textwidth}{0.5pt}

\vspace{0.5cm}

\noindent\rule[2pt]{\textwidth}{0.5pt}
%\begin{center}
%{\large\textbf{Title in english\\}}
%\end{center}
{\textbf{Abstract ---}}
The objective of this thesis is to provide new solutions to improve performance in the numerical solution of partial differential equations by studying a method for generating quadrilateral meshes based on cross-field techniques.

Our approach relies on aligning a given cross field with the boundary of the computational domain, followed by partitioning the domain into four-sided regions which are then meshed into quadrilaterals. This allows us to generate a structured block mesh while preserving the positions of the singularities of the initial cross field. To achieve this, we conduct a theoretical study of cross fields to identify the conditions under which a given cross field effectively partitions and meshes the domain on which it is defined.

This method is initially presented for simply connected domains and then extended to non-simply connected domains. It allows us to handle domains consisting of multiple materials as well as boundary singular points, which in practice delineate portions of the domain boundary to account for mixed boundary conditions in numerical simulations. Finally, we discretize the method on triangular meshes and propose a generalization to curved surfaces in space.\\


{\textbf{Keywords:}}
    Cross field, quadrilateral mesh, structured block mesh, Ginzburg Landau enery.
\\
\noindent\rule[2pt]{\textwidth}{0.5pt}
%\begin{center}
%  Nom du laboratoire, adresse du laboratoire\\
%  Ville du laboratoire
%\end{center}
\end{vcenterpage}

%%% Local Variables:
%%% mode: latex
%%% TeX-master: "../phdthesis"
%%% End:
